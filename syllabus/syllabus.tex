\documentclass[10pt]{article}
\usepackage[margin=1in]{geometry}
\usepackage{graphicx}
\graphicspath{{./images/}}

\usepackage{amsmath, amssymb, amsthm, amsfonts, mathtools, bbm, breqn}
\usepackage{enumitem}       % for shortlabels
\usepackage{multicol}       % multiple columns
\usepackage{abraces}        % asymmetric braces
\usepackage{skull}
\usepackage{tikz}
\usetikzlibrary{arrows.meta, arrows, calc, matrix, positioning}

\usepackage{hyperref}
\usepackage[capitalize]{cleveref}


% Citing theorems by name. (source: https://tex.stackexchange.com/questions/109843/cleveref-and-named-theorems)
\makeatletter
\newcommand{\ncref}[1]{\cref{#1}\mynameref{#1}{\csname r@#1\endcsname}}

\def\mynameref#1#2{%
  \begingroup
  \edef\@mytxt{#2}%
  \edef\@mytst{\expandafter\@thirdoffive\@mytxt}%
  \ifx\@mytst\empty\else
  \space(\nameref{#1})\fi
  \endgroup
}
\makeatother

% for the pipe symbol
\usepackage[T1]{fontenc}

\usepackage{xcolor} % Enables a broader range of colors

% Define custom colors
\definecolor{RoyalBlue}{cmyk}{1, 0.50, 0, 0}

% Note commands
\definecolor{Red}{rgb}{1,0,0}
\definecolor{Blue}{rgb}{0,0,1}
\definecolor{Purple}{rgb}{.75,0,.25}
\newcommand{\rnote}[1]{\textcolor{Red}{[#1]}}       % Red note
\newcommand{\pnote}[1]{\textcolor{Purple}{[#1]}}    % Purple note
\newcommand{\bnote}[1]{\textcolor{Blue}{#1}}        % Blue text

\usepackage{xparse} % allows paragraph-spaning color environment
\NewDocumentCommand{\Max}{+m}{%
  {\color{Red}#1}%
}

% Emphasized text
\newcommand{\demph}[1]{\textcolor{RoyalBlue}{\slshape #1}} % Slanted RoyalBlue text


% Claim numbering (the counter restarts after each proof environment)
\newcounter{claimcount}
\setcounter{claimcount}{0}
\newenvironment{claim}{\refstepcounter{claimcount}\par\addvspace{\medskipamount}\noindent\textbf{Claim \arabic{claimcount}:}}{}
\usepackage{etoolbox}
\AtBeginEnvironment{proof}{\setcounter{claimcount}{0}}
\newenvironment{claimproof}{\par\addvspace{\medskipamount}\noindent\textit{Proof of Claim  \arabic{claimcount}.}}{\hfill\ensuremath{\qedsymbol} \tiny{Claim}

  \medskip}
% Add claim support to cleverref
\crefname{claimcount}{Claim}{Claims}


% Math Environments
\newtheorem{theorem}{Theorem}
\newtheorem{assumption}[theorem]{Assumption}
\newtheorem{lemma}[theorem]{Lemma}
\newtheorem{proposition}[theorem]{Proposition}
\newtheorem{corollary}[theorem]{Corollary}
\newtheorem{question}[theorem]{Question}
\theoremstyle{definition}
\newtheorem{definition}[theorem]{Definition}
\newtheorem{remark}[theorem]{Remark}
\newtheorem{example}[theorem]{Example}
\newtheorem{notation}[theorem]{Notation}
\newtheorem{problem}[theorem]{Problem}

% Redefine the Example environment to include "End of example [number]"
\makeatletter
\let\oldexample\example
\renewenvironment{example}
{\begin{oldexample}}
  {\par\smallskip\hfill   End of Example~\theexample. $\square$    \par\end{oldexample}}
\makeatother

% Custom Math Commands
\newcommand{\vt}{\vspace{5mm}}                       % Vertical space
\newcommand{\fl}[1]{\noindent\textbf{#1}}            % Bold first line
\newcommand{\Fl}[1]{\vspace{5mm}\noindent\textbf{#1}}% Bold first line with space above
\newcommand{\norm}[1]{\left\lVert #1 \right\rVert}   % Norm

% Common math symbols
\newcommand{\R}{\mathbb{R}}           % Real numbers
\newcommand{\N}{\mathbb{N}}           % Natural numbers
\newcommand{\C}{\mathbb{C}}           % Complex numbers
\newcommand{\Z}{\mathbb{Z}}           % Integers
\newcommand{\Q}{\mathbb{Q}}           % Rational numbers
\newcommand{\E}{\mathbb{E}}           % Expectation
\renewcommand{\P}{\mathbb{P}}         % Probability (renamed to avoid \P clash)
\newcommand{\indep}{\perp\!\!\!\perp} % Independence symbol

% Operators
\newcommand{\Var}{\mathrm{Var}}   % Variance
\newcommand{\tr}{\mathrm{tr}}     % Trace
\newcommand{\dist}{\mathrm{dist}} % Distance


\usepackage{biblatex}
\addbibresource{refs.bib}


\begin{document}
\begin{center}
  \section*{Syllabus for Math 472: Statistical Inference}
  \subsection*{Spring 2026}
\end{center}


\section{Course description}

\subsection{Locations and Times}
\fl{Instructor:} Max Hill\\
\fl{Office Hours:} 9:00-10:00am Tuesday and Friday (or by appointment) at Physical Sciences Building 304\\
\fl{Class location and time:} Keller Hall 313 at 10:30-11:20am (MWF)\\
\fl{Course website:} \url{https://max-hill.github.io/math-472/}\\
\fl{Textbook:} \textit{Mathematical Statistics with Applications}, by
Wackerly, Mendenhall, and Scheaffer, (7th ed.~2008).


\subsection{Prerequisites}
The official prerequisites are MATH 472 or instructor consent. In practice,
you will need to have had some exposure to probability theory, proof writing,
and multivariate calculus.

\subsection{Learning objectives}
This course is intended to introduce you to the art of mathematical statistics
either in preparation for (1) further graduate-level study or (2) application
in careers in industry involving statistics, data science, or analytics. We
will focus on mathematical tools used in classical (i.e., frequentist)
statistics developed in the first half of the 20th century. At a high level,
my hope is for this course to begin a process of demystifying the techniques,
capabilities, and limitations of mathematical statistics. You will learn
standard terminology, key objects, frameworks, and methods. You will be better
at writing proofs and performing sophisticated, multi-stage computations. You
will learn the basics of the statistical software \texttt{R}.


\subsection{Tentative course outline}
\begin{enumerate}
  \item \textbf{Weeks 1-2.} Sampling distributions (4 lessons).

  chi-squared, t, and F distributions, distributions of sample mean and
  variance

  \item \textbf{Weeks 3-4.} Point estimation (5 lessons)

  properties and methods of point estimation

  \item \textbf{Weeks 5-6.} Interval estimation (4 lessons)

  Confidence intervals for means, variances, proportions and differences

  \item \textbf{Weeks 7-12.} Hypothesis Testing (19 lessons)

  Neyman-Pearson lemma, likelihood ratio test; tests concerning means and
  variances, tests based on count data, nonparametric tests, analysis of variance

  \item \textbf{Weeks 13-14.} Regression and correlation (6 lessons)

  regression, bivariate normal distributions, method of least squares
\end{enumerate}

\section{Important Dates}

Non-instructional days are January 19, February 16, March 16-20.

\begin{center} \begin{minipage}{5.0in}
    \begin{flushleft}
      Midterm 1 \dotfill~Wednesday, Feb 18 (in class) \\
      Midterm 2 \dotfill~Friday, March 27 (in class) \\
      Final exam \dotfill~Friday, May 15 at 9:45-11:45am\\
    \end{flushleft}
  \end{minipage}
\end{center}
The last day of instruction is May 6.



\section{Grading}
Final grades will be computed as a weighted average of the following three
categories:

\begin{itemize}
  \item \textbf{homework (20\%)}
  Homeworks will be due approximately weekly. I will incorporate coding elements
  into the homework, using the statistical software \texttt{R}. This is the most
  widely-used statistical software. This is something you will be able to put on
  your cv/resume.

  You have one `no questions asked' homework estension. 

  \item \textbf{quizzes (10\%)}
  There will be regular in-class quizzes in which I ask you to state precise
  mathematical definitions. These quizzes will be only a few minutes.

  You can retake up to two quizzes in office hours.

  \item \textbf{exams (70\%)}
  There are two midterms (17.5\% each) and one final exam (35\%). All exams will
  be closed-book, and my include some homework problems. The final exam will be
  cumulative.

  Make-up exams are allowed only in three types of circumstances: (1) in
  accordance with university policies, such as conflict with a religious
  observation, (2) conflicts with another university-related event, or (3)
  exceptional circumstances, such as a last-minute medical or family emergency
  with verification. In the first two cases, notice must be given to the
  instructor two weeks in advance.
\end{itemize}
The following grade cutoffs will be used at
the end of the semester to determine final grades:
\begin{center}
  \begin{tabular}{|c|c|c|c|c|c|c|c|c|c|c|}
    \hline
    D&D+&C-&C&C+&B-&B&B+&A-&A&A+  \\
    \hline
    60\%&67\%&70\%&73\%&77\%&80\%&83\%&87\%&90\%&93\%&97\%\\
    \hline
  \end{tabular}
\end{center}

\section{Other policies}

\subsection{Collaboration and outside resources}

You may collaborate with classmates on the homeworks. But if you do so, you
must (1) make an effort write up your solutions on your own, using your own
words, and (2) list the names of those who you worked with.

Use of outside resources is allowed but must be cited, in the sense that you
need to tell me what you used and how you used it. I do not know how best to
use AI tools to increase learning, so this is something I'd like you to think
about over the course of the semester, and give me feedback if you have any.

\subsection{Incompletes}
An incomplete is possible only if all of the following apply: (1) you have a
compelling personal reason, e.g., serious illness or accident (a proof, e.g.,
report from a doctor or police must be shown); (2) your work so far would
receive a passing grade; and (3) there is a good chance you will complete the
course with a passing grade within the allotted time. Thus, expecting to fail
the class is not a reason to ask for an incomplete.

\subsection{Accommodation statement}
The University of Hawai'i, M\={a}noa is committed to providing an equal
educational opportunity for all students. A student with a documented
physical, psychological, or learning disability on file with KOKUA Program
(Disability Access Services, \url{http://www.hawaii.edu/kokua/}) may be
eligible for reasonable academic accommodations to help succeed in this
course. If you have a documented disability that requires an accommodation,
please ask KOKUA office to notify the instructor within the first two weeks of
the semester in order to make appropriate arrangements.





\printbibliography
\end{document}