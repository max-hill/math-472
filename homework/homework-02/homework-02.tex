\documentclass[10pt]{article}
% Math Packages
\usepackage{amsmath, mathtools}
\usepackage{amssymb}
\usepackage{amsthm}
\usepackage{amsfonts}
\usepackage{bbm}
\usepackage{breqn}
\usepackage[margin=1in]{geometry}
\usepackage{graphicx}
\usepackage{tikz}
\usetikzlibrary{arrows.meta}
\usetikzlibrary{calc}
\usepackage{forest}
\usepackage{tikz-qtree}
\graphicspath{ {./images/} }
\usepackage{hyperref}
\usepackage[capitalize]{cleveref}
\usepackage[shortlabels]{enumitem}
\usetikzlibrary{arrows,matrix,positioning}
\usepackage{multicol}

\usepackage{listings}
\usepackage{xcolor}

\lstset{
  language=R,
  basicstyle=\ttfamily\footnotesize,
  keywordstyle=\color{blue},
  commentstyle=\color{gray},
  stringstyle=\color{red!70!black},
  framesep=2pt,
  framexleftmargin=0pt, 
  frame=single,
  breaklines=false,
  showstringspaces=false
}
% for the pipe symbol
\usepackage[T1]{fontenc}

% Citing theorems by name. (source: https://tex.stackexchange.com/questions/109843/cleveref-and-named-theorems)
\makeatletter
\newcommand{\ncref}[1]{\cref{#1}\mynameref{#1}{\csname r@#1\endcsname}}

\def\mynameref#1#2{%
  \begingroup
  \edef\@mytxt{#2}%
  \edef\@mytst{\expandafter\@thirdoffive\@mytxt}%
  \ifx\@mytst\empty\else
  \space(\nameref{#1})\fi
  \endgroup
}
\makeatother

% Colorful Notes
\usepackage{color} \definecolor{Red}{rgb}{1,0,0} \definecolor{Blue}{rgb}{0,0,1}
\definecolor{Purple}{rgb}{.5,0,.5} \def\red{\color{Red}} \def\blue{\color{Blue}}
\def\gray{\color{gray}} \def\purple{\color{Purple}}
\newcommand{\rnote}[1]{{\red [#1]}} % \rnote{foo} gives '[foo]' in red
\newcommand{\pnote}[1]{{\purple [#1]}} % \pnote{foo} gives '[foo]' in purple
\newcommand{\bnote}[1]{{\blue #1}} % \bnote{foo} gives 'foo' in blue
\newcommand{\gnote}[1]{{\gray #1}} % \gnote{foo} gives 'foo' in gray
\newcommand{\Max}[1]{{\purple [#1]}} % \bnote{foo} then 'foo' is blue


% Claim numbering (the counter restarts after each proof environment)
\newcounter{claimcount}
\setcounter{claimcount}{0}
\newenvironment{claim}{\refstepcounter{claimcount}\par\addvspace{\medskipamount}\noindent\textbf{Claim \arabic{claimcount}:}}{}
\usepackage{etoolbox}
\AtBeginEnvironment{proof}{\setcounter{claimcount}{0}}
\newenvironment{claimproof}{\par\addvspace{\medskipamount}\noindent\textit{Proof of Claim  \arabic{claimcount}.}}{\hfill\ensuremath{\qedsymbol} \tiny{Claim}

  \medskip}
% Add claim support to cleverref
\crefname{claimcount}{Claim}{Claims}


% Math Environments
\newtheorem{theorem}{Theorem}
\newtheorem{assumption}[theorem]{Assumption}
\newtheorem{lemma}[theorem]{Lemma}
\newtheorem{proposition}[theorem]{Proposition}
\newtheorem{corollary}[theorem]{Corollary}
\newtheorem{question}[theorem]{Question}
\theoremstyle{definition}
\newtheorem{definition}[theorem]{Definition}
\newtheorem{remark}[theorem]{Remark}
\newtheorem{example}[theorem]{Example}
\newtheorem{notation}[theorem]{Notation}
\newtheorem{problem}[theorem]{Problem}

% Matrices and Column Vectors. 
\usepackage{stackengine}
\setstackgap{L}{1.0\normalbaselineskip}
\usepackage{tabstackengine}
\setstackEOL{;}% row separator
\setstackTAB{,}% column separator
\setstacktabbedgap{1ex}% inter-column gap
\setstackgap{L}{1.5\normalbaselineskip}% inter-row baselineskip
\let\nmatrix\bracketMatrixstack  %Usage: \nmatrix{1,2,3\4,5,6}
\newcommand\cv[1]{\setstackEOL{,}\bracketMatrixstack{#1}} %usage: \cv{1,2,3}

% Custom Math Coqmmands
\newcommand{\vt}{\vskip 5mm} % vertical space
\newcommand{\fl}{\noindent\textbf} % first line
\newcommand{\Fl}{\vt\noindent\textbf} % first line with space above
\newcommand{\norm}[1]{\left\lVert#1\right\rVert} % norm
\newcommand{\pnorm}[1]{\left\lVert#1\right\rVert_p} % p-norm
\newcommand{\qnorm}[1]{\left\lVert#1\right\rVert_q} % q-norm
\newcommand{\1}[1]{\textbf{1}_{\left[#1\right]}} % indicator function 
\def\limn{\lim_{n\to\infty}} % shortcut for lim as n-> infinity
\def\sumn{\sum_{n=1}^{\infty}} % shortcut for sum from n=1 to infinity
\def\sumkn{\sum_{k=1}^{n}} % shortcut for sum from k=1 to n
\def\sumin{\sum_{i=1}^{n}} % shortcut for sum from i=1 to n
\def\SAs{\sigma\text{-algebras}} % shortcut for $\sigma$-algebras
\def\SA{\sigma\text{-algebra}} % shortcut for $\sigma$-algebra
\def\Ft{\mathcal{F}_t} % time-indexed sigma-algebra (t)
\def\Fs{\mathcal{F}_s} % time-indexed sigma-algebra (s)
\def\F{\mathcal{F}} % sigma-algebra
\def\G{\mathcal{G}} % sigma-algebra
\def\R{\mathbb{R}} % Real numbers
\def\N{\mathbb{N}} % Natural numbers
\def\Z{\mathbb{Z}} % Integers
\def\E{\mathbb{E}} % Expectation
\def\P{\mathbb{P}} % Probability
\def\Q{\mathbb{Q}} % Q probability
\def\dist{\text{dist}} %Text 'dist' for things like 'dist(x,y)'
\newcommand{\indep}{\perp \!\!\! \perp}  %independence symbol
\def\Var{\mathrm{Var}} % Variance
\def\tr{\mathrm{tr}} % trace

% Brackets and Parentheses
\def\[{\left [}
    \def\]{\right ]}
% \def\({\left (}
%   \def\){\right )}



\usepackage{color}
\definecolor{Red}{rgb}{1,0,0}
\definecolor{Blue}{rgb}{0,0,1}
\definecolor{Purple}{rgb}{.5,0,.5}
\def\red{\color{Red}}
\def\blue{\color{Blue}}
\def\gray{\color{gray}}
\def\purple{\color{Purple}}
\definecolor{RoyalBlue}{cmyk}{1, 0.50, 0, 0}
\newcommand{\dempfcolor}[1]{{\color{RoyalBlue}#1}} 
\newcommand{\demph}[1]{\textcolor{RoyalBlue}{\textbf{\slshape #1}}} % Slanted

\usepackage{emoji}
% comment exactly one of the following line to show / hide the solutions
% \newcommand{\solution}[1]{{\purple #1}} % uncomment to show the solutions
\newcommand{\solution}[1]{} % uncomment to hide the solutions




\begin{document}
\begin{center}
  \section*{Math 472: Homework 02} %
  \textit{Due Friday, Jan 30}
\end{center}





% \begin{problem}
%   \begin{enumerate}[(a)]
%     Let $x\in \R$ be arbitrary.
%     \item Prove that 
%     \begin{equation*}
%       \lim_{n\to\infty} \left( 1+ \frac{x}{n} \right)^{n} = e^{x} 
%     \end{equation*}
%     \item Suppose that $(x_{n})$ is a sequence converging to $x$. Prove that
%     \begin{equation*}
%       \lim_{n\to\infty} \left( 1+\frac{x_{n}}{n} \right)^{n}=e^{x}. 
%     \end{equation*}
%   \end{enumerate}
% \end{problem}




\begin{problem}
  Prove \textbf{Proposition 2} in the lecture notes (``Basic properties of probability measures'').
  
  \textit{If you get stuck, refer to my old math 372 lecture notes (Week 2,
    Class 4, available on the course website).}
\end{problem}


\begin{problem}
  In lecture, we noted that the variance of a sample mean decreases as the
  size of the sample increases. In this problem, we'll visualize this
  phenomenon for dice rolls like those of Example 7.1 in the textbook.
  \begin{enumerate}[(a)]
    \item For any positive integer $n$, the \texttt{R} code

    \texttt{mean(sample(1:6, n, replace=TRUE))}

    simulates rolling a 6-sided dice $n$ times. This is our sample. Write code
    to compute the sample mean of this sample.
    \item For each $k\in \left\{3,10,100,1000\right\}$, use the functions
    \texttt{replicate()} and \texttt{hist()} to plot a histogram consisting of
    10,000 sample means (obtained using your code from part (a)).

    When plotting the histograms with the \texttt{hist()} function, add the
    optional argument \texttt{xlim=c(0,6)} and observe how the histograms get
    narrower and more concentrated around 3.5. That's the whole point of this
    problem: the variance of a sample mean tends to zero as $n \to \infty$.
    
    \item In part (b), we visualized the variance of the \textit{sample mean}
    through simulations. In this part, we will consider the \textit{sample
      variance}, which is a different quantity. For
    \begin{equation*}
      n\in \left\{10,30,60,100,1000,10000,100000,500000,1000000,10000000,100000000\right\},
    \end{equation*}
    draw a sample of $n$ dice rolls and compute the sample variance $S^{2}$.
    What quantity does this appear to be converging to?
  \end{enumerate}
\end{problem}
\begin{problem}
  An example of a random variable whose expected value does not exist is the
  \demph{Cauchy random variable}, that is, one with pdf
  \begin{equation*}
    f(x) = \frac{1}{\pi} \frac{1}{1+x^{2}}, \ x\in \R.
  \end{equation*}
  \begin{enumerate}[(a)]
    \item Show that $f$ is a valid probability density function by showing
    that it is nonnegative and that $\int_{-\infty}^{\infty}f(x)dx =1.$ 
    \item  Show that $\E\left[X \right]$ is not defined by showing that
    $E \left[ |X| \right] = \int_{-\infty}^{\infty}|x|f(x)dx  =+\infty$.
  \end{enumerate}
\end{problem}

\begin{problem}
  A ``median'' of a distribution is a value $m$ such that $\P\left[X \leq
    m\right] \geq \frac{1}{2}$ and $\P\left[X \geq m\right] \geq \frac{1}{2}$.
  (If $X$ is continuous, $m$ satisfies $\int_{-\infty}^{m}f(x)dx =
  \int_{m}^{\infty}f(x)dx=\frac{1}{2}$, where $f$ is the pdf of $X$.) Find the
  median of the following distributions
  \begin{enumerate}[(a)]
    \item $f(x)=3x^{2},\ 0<x<1$
    \item $f(x)= \frac{1}{\pi} \frac{1}{1+x^{2}},\ x\in \R$.
  \end{enumerate}
\end{problem}

\begin{problem}[Exercise 2.18 in textbook]
  Suppose two fair coins are tossed and the upper faces are observed.
  \begin{enumerate}[(a)]
    \item List the sample points for this experiment.
    \item Assign a reasonable probability to each sample point. (Are the sample
    points equally likely?)
    \item Let $A$ denote the event that exactly one head is observed and $B$ the
    event that at least one head is observed. List the sample points in both $A$ 
    and $B$. 
    \item From your answer to part (c), find $P(A)$, $P(B)$, $P(A\cap B)$, $P(A\cup B)$, and
    $P(A \cup B)$. 
  \end{enumerate}
\end{problem}




\begin{problem}[Variance]
  \label{problem:variance}
  The \demph{variance} of a random variable $X$ is the quantity
  \begin{equation*}
    \Var(X) = \E\left[\left( X- \mu \right)^{2}  \right],
  \end{equation*}
  where $\mu= \E\left[X \right]$. The positive square root $\Var(X)$ is called
  the \demph{standard deviation} of $X$.

  In this problem, we'll prove three important facts about variance.
  \begin{enumerate}[(a)]
    \item Prove that 
    \begin{equation*}
      \Var(X) = \E\left[X^{2} \right] - \left( \E\left[X \right] \right)^{2}. 
    \end{equation*}
    \item Prove that if $X$ is a random variable, then for any scalars $a$ and
    $b$,
    \begin{equation}\label{eq:1}
      \Var(aX+b) = a^{2}\Var(X).
    \end{equation}
    \item If $X$ and $Y$ are independent random variables, 
    \begin{equation}\label{eq:2}
      \Var(X+ Y) = \Var(X)+\Var(Y).
    \end{equation}
    \textit{Hint: Since $X$ and $Y$ are independent,
     $\E\left[\left(
        X- \E\left[X \right]  \right) \left( Y - \E\left[Y \right] \right)
    \right]= 0$.}
  \end{enumerate}
\end{problem}



\begin{problem}
  Let $A$ and $B$ be independent events. Show that the following pairs of
  events are independent.
  \begin{enumerate}[(a)]
    \item $A$ and $B^{c}$
    \item $A^{c}$ and $B$
    \item $A^{c}$ and $B^{c}$.
  \end{enumerate}
\end{problem}


% \begin{problem}[The weak law of large numbers]
%   A coin has probability $p$ of landing on heads and probability $1-p$ of
%   landing on tails. Suppose we start flipping the coin; for each positive
%   integer $i$ define
%   \begin{equation*}
%     X_{i} = \left\{ \begin{array}{l@{\quad:\quad}l} 1 & \text{ if the $i^{\rm
%             th}$ flip is heads }\\ 0& \text{ if the $i^{\rm th}$ flip is tails} \end{array}\right.
%   \end{equation*}
%   Assume that $X_{1},X_{2},\ldots,$ are independent. Then
%   $S_{n}=X_{1}+\ldots+X_{n}$ is the \textit{number} of heads in the first $n$
%   flips, and $\frac{S_{n}}{n}$ is the \textit{proportion} of heads in the
%   first $n$ flips.
  
  
%   \begin{enumerate}[(a)]
%     \item Show that $\E\left[\frac{S_{n}}{n} \right]= p$.
%     \item Show that $\Var(\frac{S_{n}}{n}) = \frac{p(1-p)}{n}$. \textit{(Hint:
%       use \cref{eq:1,eq:2} from \cref{problem:variance}).}
%     \item \label{item:1} Let $\epsilon>0$ be arbitrary. Using
%     Chebychev's inequality and the squeeze theorem, show that
%     \begin{equation*}
%       \lim_{n\to\infty} \P\left[\left| \frac{S_{n}}{n}-p \right| >\epsilon\right] = 0.
%     \end{equation*}
%     \item Using one or two complete sentences, intepret the limit in part
%     \ref{item:1} when $\epsilon$ is very small.
%   \end{enumerate}
% \end{problem}


\begin{problem}
  Let $X$ be a random variable with pdf $f$ and cdf $F$. Assume that $f$ is uniformly
  bounded, i.e., that there exists $M>0$ such that $|f(t)| \leq M$ for all
  $t\in \R$.
  \begin{enumerate}[(a)]
    \item Prove that if $x,y\in \R$ then $\left| F(x)-F(y) \right| \leq M \left| x-y
    \right|$.
    \item Use part (a) to show that $F$ is continuous on $\R$.
  \end{enumerate}
\end{problem}


\begin{problem}[Exercise 1.9 in the textbook]
  Resting breathing rates for college-age students are approximately normally distributed with
  mean 12 and standard deviation 2.3 breaths per minute. What fraction of all college-age students
  have breathing rates in the following intervals?
  \begin{enumerate}[(a)]
    \item   9.7 to 14.3 breaths per minute
    \item   7.4 to 16.6 breaths per minute
    \item   9.7 to 16.6 breaths per minute
    \item   Less than 5.1 or more than 18.9 breaths per minute
  \end{enumerate}
\end{problem}

\end{document}