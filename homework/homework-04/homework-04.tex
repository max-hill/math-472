\documentclass[10pt]{article}
% Math Packages
\usepackage{amsmath, mathtools}
\usepackage{amssymb}
\usepackage{amsthm}
\usepackage{amsfonts}
\usepackage{bbm}
\usepackage{breqn}
\usepackage[margin=1in]{geometry}
\usepackage{graphicx}
\usepackage{tikz}
\usetikzlibrary{arrows.meta}
\usetikzlibrary{calc}
\usepackage{forest}
\usepackage{tikz-qtree}
\graphicspath{ {./images/} }
\usepackage{hyperref}
\usepackage[capitalize]{cleveref}
\usepackage[shortlabels]{enumitem}
\usetikzlibrary{arrows,matrix,positioning}
\usepackage{multicol}

\usepackage{listings}
\usepackage{xcolor}

\crefname{problem}{Problem}{Problems}
\Crefname{problem}{Problem}{Problems}

\lstset{
  language=R,
  basicstyle=\ttfamily\footnotesize,
  keywordstyle=\color{blue},
  commentstyle=\color{gray},
  stringstyle=\color{red!70!black},
  framesep=2pt,
  framexleftmargin=0pt, 
  frame=single,
  breaklines=false,
  showstringspaces=false
}
% for the pipe symbol
\usepackage[T1]{fontenc}

% Citing theorems by name. (source: https://tex.stackexchange.com/questions/109843/cleveref-and-named-theorems)
\makeatletter
\newcommand{\ncref}[1]{\cref{#1}\mynameref{#1}{\csname r@#1\endcsname}}

\def\mynameref#1#2{%
  \begingroup
  \edef\@mytxt{#2}%
  \edef\@mytst{\expandafter\@thirdoffive\@mytxt}%
  \ifx\@mytst\empty\else
  \space(\nameref{#1})\fi
  \endgroup
}
\makeatother

% Colorful Notes
\usepackage{color} \definecolor{Red}{rgb}{1,0,0} \definecolor{Blue}{rgb}{0,0,1}
\definecolor{Purple}{rgb}{.5,0,.5} \def\red{\color{Red}} \def\blue{\color{Blue}}
\def\gray{\color{gray}} \def\purple{\color{Purple}}
\newcommand{\rnote}[1]{{\red [#1]}} % \rnote{foo} gives '[foo]' in red
\newcommand{\pnote}[1]{{\purple [#1]}} % \pnote{foo} gives '[foo]' in purple
\newcommand{\bnote}[1]{{\blue #1}} % \bnote{foo} gives 'foo' in blue
\newcommand{\gnote}[1]{{\gray #1}} % \gnote{foo} gives 'foo' in gray
\newcommand{\Max}[1]{{\purple [#1]}} % \bnote{foo} then 'foo' is blue


% Claim numbering (the counter restarts after each proof environment)
\newcounter{claimcount}
\setcounter{claimcount}{0}
\newenvironment{claim}{\refstepcounter{claimcount}\par\addvspace{\medskipamount}\noindent\textbf{Claim \arabic{claimcount}:}}{}
\usepackage{etoolbox}
\AtBeginEnvironment{proof}{\setcounter{claimcount}{0}}
\newenvironment{claimproof}{\par\addvspace{\medskipamount}\noindent\textit{Proof of Claim  \arabic{claimcount}.}}{\hfill\ensuremath{\qedsymbol} \tiny{Claim}

  \medskip}
% Add claim support to cleverref
\crefname{claimcount}{Claim}{Claims}


% Math Environments
\newtheorem{theorem}{Theorem}
\newtheorem{assumption}[theorem]{Assumption}
\newtheorem{lemma}[theorem]{Lemma}
\newtheorem{proposition}[theorem]{Proposition}
\newtheorem{corollary}[theorem]{Corollary}
\newtheorem{question}[theorem]{Question}
\theoremstyle{definition}
\newtheorem*{definition}{Definition}
\newtheorem{remark}[theorem]{Remark}
\newtheorem{example}[theorem]{Example}
\newtheorem{notation}[theorem]{Notation}
\newtheorem{problem}[theorem]{Problem}

% Matrices and Column Vectors. 
\usepackage{stackengine}
\setstackgap{L}{1.0\normalbaselineskip}
\usepackage{tabstackengine}
\setstackEOL{;}% row separator
\setstackTAB{,}% column separator
\setstacktabbedgap{1ex}% inter-column gap
\setstackgap{L}{1.5\normalbaselineskip}% inter-row baselineskip
\let\nmatrix\bracketMatrixstack  %Usage: \nmatrix{1,2,3\4,5,6}
\newcommand\cv[1]{\setstackEOL{,}\bracketMatrixstack{#1}} %usage: \cv{1,2,3}

% Custom Math Coqmmands
\newcommand{\vt}{\vskip 5mm} % vertical space
\newcommand{\fl}{\noindent\textbf} % first line
\newcommand{\Fl}{\vt\noindent\textbf} % first line with space above
\newcommand{\norm}[1]{\left\lVert#1\right\rVert} % norm
\newcommand{\pnorm}[1]{\left\lVert#1\right\rVert_p} % p-norm
\newcommand{\qnorm}[1]{\left\lVert#1\right\rVert_q} % q-norm
\newcommand{\1}[1]{\textbf{1}_{\left[#1\right]}} % indicator function 
\def\limn{\lim_{n\to\infty}} % shortcut for lim as n-> infinity
\def\sumn{\sum_{n=1}^{\infty}} % shortcut for sum from n=1 to infinity
\def\sumkn{\sum_{k=1}^{n}} % shortcut for sum from k=1 to n
\def\sumin{\sum_{i=1}^{n}} % shortcut for sum from i=1 to n
\def\SAs{\sigma\text{-algebras}} % shortcut for $\sigma$-algebras
\def\SA{\sigma\text{-algebra}} % shortcut for $\sigma$-algebra
\def\Ft{\mathcal{F}_t} % time-indexed sigma-algebra (t)
\def\Fs{\mathcal{F}_s} % time-indexed sigma-algebra (s)
\def\F{\mathcal{F}} % sigma-algebra
\def\G{\mathcal{G}} % sigma-algebra
\def\R{\mathbb{R}} % Real numbers
\def\N{\mathbb{N}} % Natural numbers
\def\Z{\mathbb{Z}} % Integers
\def\E{\mathbb{E}} % Expectation
\def\P{\mathbb{P}} % Probability
\def\Q{\mathbb{Q}} % Q probability
\def\dist{\text{dist}} %Text 'dist' for things like 'dist(x,y)'
\newcommand{\indep}{\perp \!\!\! \perp}  %independence symbol
\def\Var{\mathrm{Var}} % Variance
\def\tr{\mathrm{tr}} % trace

% Brackets and Parentheses
\def\[{\left [}
    \def\]{\right ]}
% \def\({\left (}
%   \def\){\right )}



\usepackage{color}
\definecolor{Red}{rgb}{1,0,0}
\definecolor{Blue}{rgb}{0,0,1}
\definecolor{Purple}{rgb}{.5,0,.5}
\def\red{\color{Red}}
\def\blue{\color{Blue}}
\def\gray{\color{gray}}
\def\purple{\color{Purple}}
\definecolor{RoyalBlue}{cmyk}{1, 0.50, 0, 0}
\newcommand{\dempfcolor}[1]{{\color{RoyalBlue}#1}} 
\newcommand{\demph}[1]{\textcolor{RoyalBlue}{\textbf{\slshape #1}}} % Slanted

\usepackage{emoji}
% comment exactly one of the following line to show / hide the solutions
% \newcommand{\solution}[1]{{\purple #1}} % uncomment to show the solutions
\newcommand{\solution}[1]{} % uncomment to hide the solutions




\begin{document}
\begin{center}
  \section*{Math 472: Homework 04}
  \textit{Due Monday Feb 16}
\end{center}

\begin{problem}[Geometric distribution]
  Consider a sequence of independent coin flips, each of which has probability
  $p$ of being heads. Define a random variable $X$ as the length of the run
  (either heads or tails) started by the first trial. (For example, if
  \texttt{TTTH} or \texttt{HHHT} is observed).
  \begin{enumerate}[(a)]
    \item What is the sample space? What are the sample points? 
    \item Find the distribution of $X$ (i.e., compute the probability mass
    function of $X$).
    \item Find $\E\left[X \right]$.
    \item Use \texttt{R} to generate 50 samples from the distribution of $X$
    (use whatever value of $p\in (0,1)$ you want). What is the sample mean?
    What is the sample variance?
  \end{enumerate}
\end{problem}



For \cref{problem:1,problem:2,problem:3,problem:4,problem:5}, the following
plot may be helpful:

\begin{center}
  \includegraphics[scale=0.3]{images/standard-normal-small}
\end{center}



\begin{problem}%[CLT]
  \label{problem:2}
  Let $S$ be the number of heads in 100 tosses of a fair coin. Use the
  Central Limit Theorem to compute $\P\left[45 \leq S\leq 55\right]$.
  \textit{Your answer should be a number.}
\end{problem}


\begin{problem}%[CLT]
  \label{problem:3}
  Use the Central Limit Theorem to approximate the probability of obtaining
  more than 65 heads when flipping a fair coin 100 times.
\end{problem}

\begin{problem}%[CLT] % similar to hw 9, problem 6
  \label{problem:1}
  The nation of Oceania has always been at war with the nation of Eurasia. To
  establish peace, Oceania's leadership has unanimously authorized a
  preemptive strike to be carried out tomorrow morning against 48 pre-selected
  military and industrial targets in enemy territory. Analysts at the Ministry
  of Peace estimate that each target has a probability of $3/4$ of being
  destroyed in this first strike.

  The operation will be regarded as a strategic success under Oceania's peace
  doctrine if at least 30 targets are destroyed. Using the Central Limit
  Theorem, estimate the probability that the first strike achieves this
  benchmark.\footnote{Responses deviating significantly from 100\% confidence
    will be flagged for review by the Ministry of Truth.}

\end{problem}




\begin{problem}%[CLT: this problem is similar to problem 9 in homework 9]
  \label{problem:4}
  A mandolin-making machine in Mordecai's mandolin manufactory makes about 5\%
  defective mandolins even when properly functioning. The mandolins are then
  packed in crates containing 1900 mandolins each. A crate is examined and
  found to contain 115 defective mandolins. What is the approximate
  probability of finding at least this many defective mandolins if the machine
  is properly adjusted? If you were Mordecai, would hire a technician to check
  out the machine?

  %\textit{Hint:} The following may be useful $\sqrt{1900}=10 \sqrt{19}$.
\end{problem}


\begin{problem}
  \label{problem:5}
  A candidate believes that she can win a city election if she can earn at
  least $55\%$ of the votes in precinct 1. She also believes that about $50\%$
  of the city's voters favor her. If $n=100$ voters show up to vote at precinct
  1, what is the probability that she will recieve at least 55\% of their
  votes?
\end{problem}

\begin{problem}[Exercises 8.2, 8.3, and 8.4 in the textbook]
  Suppose $\widehat{\theta}$ is an estimator for a target parameter $\theta$.
  \begin{enumerate}[(a)]
    \item If $\widehat{\theta}$ is unbiased, what is $\text{Bias}(\widehat{\theta})$?
    \item If $\text{Bias}(\widehat{\theta})=5$, what is $\E[\widehat{\theta} ]$?
    \item Suppose $\E[\widehat{\theta} ]=a\theta+b$ for some
    nonzero constants $a$ and $b$. In terms of $a$ and $b$, what is
    $\text{Bias}(\widehat{\theta})$? Find a function of
    $\widehat{\theta}$---say, $\widehat{\theta}^{*}$---that is an unbiased
    estimator for $\theta$.
    \item If $\widehat{\theta}$ is unbiased, how does
    $\text{MSE}(\widehat{\theta})$ compare to $\text{Var}(\widehat{\theta})$?
    What about when $\widehat{\theta}$ is biased?
  \end{enumerate}
\end{problem}

\begin{problem}[Exercise 8.8 in the textbook]
  Suppose that $Y_{1},Y_{2},Y_{3}$ are a random sample from an exponential
  distribution with density function
  \begin{equation*}
    f(y) = \frac{1}{\theta}e^{-y/\theta}\mathbf{1}_{\left[ y>0 \right] }
  \end{equation*}
  Consider the following estimators
  \begin{equation*}
    \hat{\theta}_{1} =Y_{1}, \quad\quad \widehat{\theta}_{2}= \frac{Y_{1}+Y_{2}}{2}, \quad\quad \widehat{\theta}_{3} = \frac{Y_{1}+2Y_{2}}{3}, \quad\quad \hat{\theta}_{4} = \min(Y_{1},Y_{2},Y_{3}), \quad\quad \hat{\theta}_{5}= \overline{Y}
  \end{equation*}
  (Hint: recall the fact that the minimum of exponential random variables is
  itself exponentially distributed.).
  \begin{enumerate}[(a)]
    \item Which of these estimators is unbiased? 
    \item Among the unbiased estimators, which has the smallest variance?
  \end{enumerate}
\end{problem}

\begin{problem}[Exercise 8.12 in the textbook]
  The reading on a voltage meter connected to a test circuit is uniformly
  distributed over the interval $(\theta,\theta+1)$, where $\theta$ is the
  true but unknown voltage of the circuit. Suppose that $Y_{1},\ldots,Y_{n}$
  denote a random sample of such readings
  \begin{enumerate}[(a)]
    \item Show that $\overline{Y}$ is a biased estimator of $\theta$ and
    compute the bias.
    \item Find a function of $\overline{Y}$ that is an unbiased estimator of
    $\theta$. 
    \item Find $\text{MSE}(\overline{Y})$ when $\overline{Y}$ is used as an
    estimator of $\theta$. 
  \end{enumerate}
\end{problem}

\begin{problem}[Exercise 8.23 in the textbook -- similar to textbook Example 8.2]
  The Environmental Protection Agency and the University of Florida recently
  cooperated in a large study of the possible effects of trace elements in
  drinking water on kidney-stone disease. The accompanying table presents data
  on age, amount of calcium in home drinking water (measured in parts per
  million), and smoking activity. These data were obtained from individ- uals
  with recurrent kidney-stone problems, all of whom lived in the Carolinas and
  the Rocky Mountain states.

  \begin{table}[h]
    \centering
    \begin{tabular}{lcc}
      \hline
      & Carolinas & Rockies \\
      \hline
      Sample size & 467 & 191 \\
      Mean age & 45.1 & 46.4 \\
      Standard deviation of age & 10.2 & 9.8 \\
      Mean calcium component (ppm) & 11.3 & 40.1 \\
      Standard deviation of calcium & 16.6 & 28.4 \\
      Proportion now smoking & 0.78 & 0.61 \\
      \hline
    \end{tabular}
  \end{table}
  \begin{enumerate}[(a)]
    \item Estimate the average calcium concentration in drinking water for kidney-stone patients in
    the Carolinas. Place a bound on the error of estimation.
    \item Estimate the difference in mean ages for kidney-stone patients in the Carolinas and in the
    Rockies. Place a bound on the error of estimation.
    \item Estimate and place a 2-standard-deviation bound on the difference in
    proportions of kidney-stone patients from the Carolinas and Rockies who were
    smokers at the time of the study.
  \end{enumerate}
\end{problem}



\end{document}