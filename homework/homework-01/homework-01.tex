\documentclass[10pt]{article}
% Math Packages
\usepackage{amsmath, mathtools}
\usepackage{amssymb}
\usepackage{amsthm}
\usepackage{amsfonts}
\usepackage{bbm}
\usepackage{breqn}
\usepackage[margin=1in]{geometry}
\usepackage{graphicx}
\usepackage{tikz}
\usetikzlibrary{arrows.meta}
\usetikzlibrary{calc}
\usepackage{forest}
\usepackage{tikz-qtree}
\graphicspath{ {./images/} }
\usepackage{hyperref}
\usepackage[capitalize]{cleveref}
\usepackage[shortlabels]{enumitem}
\usetikzlibrary{arrows,matrix,positioning}
\usepackage{multicol}


% for the pipe symbol
\usepackage[T1]{fontenc}

% Citing theorems by name. (source: https://tex.stackexchange.com/questions/109843/cleveref-and-named-theorems)
\makeatletter
\newcommand{\ncref}[1]{\cref{#1}\mynameref{#1}{\csname r@#1\endcsname}}

\def\mynameref#1#2{%
  \begingroup
  \edef\@mytxt{#2}%
  \edef\@mytst{\expandafter\@thirdoffive\@mytxt}%
  \ifx\@mytst\empty\else
  \space(\nameref{#1})\fi
  \endgroup
}
\makeatother

% Colorful Notes
\usepackage{color} \definecolor{Red}{rgb}{1,0,0} \definecolor{Blue}{rgb}{0,0,1}
\definecolor{Purple}{rgb}{.5,0,.5} \def\red{\color{Red}} \def\blue{\color{Blue}}
\def\gray{\color{gray}} \def\purple{\color{Purple}}
\newcommand{\rnote}[1]{{\red [#1]}} % \rnote{foo} gives '[foo]' in red
\newcommand{\pnote}[1]{{\purple [#1]}} % \pnote{foo} gives '[foo]' in purple
\newcommand{\bnote}[1]{{\blue #1}} % \bnote{foo} gives 'foo' in blue
\newcommand{\gnote}[1]{{\gray #1}} % \gnote{foo} gives 'foo' in gray
\newcommand{\Max}[1]{{\purple [#1]}} % \bnote{foo} then 'foo' is blue


% Claim numbering (the counter restarts after each proof environment)
\newcounter{claimcount}
\setcounter{claimcount}{0}
\newenvironment{claim}{\refstepcounter{claimcount}\par\addvspace{\medskipamount}\noindent\textbf{Claim \arabic{claimcount}:}}{}
\usepackage{etoolbox}
\AtBeginEnvironment{proof}{\setcounter{claimcount}{0}}
\newenvironment{claimproof}{\par\addvspace{\medskipamount}\noindent\textit{Proof of Claim  \arabic{claimcount}.}}{\hfill\ensuremath{\qedsymbol} \tiny{Claim}

  \medskip}
% Add claim support to cleverref
\crefname{claimcount}{Claim}{Claims}


% Math Environments
\newtheorem{theorem}{Theorem}
\newtheorem{assumption}[theorem]{Assumption}
\newtheorem{lemma}[theorem]{Lemma}
\newtheorem{proposition}[theorem]{Proposition}
\newtheorem{corollary}[theorem]{Corollary}
\newtheorem{question}[theorem]{Question}
\theoremstyle{definition}
\newtheorem{definition}[theorem]{Definition}
\newtheorem{remark}[theorem]{Remark}
\newtheorem{example}[theorem]{Example}
\newtheorem{notation}[theorem]{Notation}
\newtheorem{problem}[theorem]{Problem}

% Matrices and Column Vectors. 
\usepackage{stackengine}
\setstackgap{L}{1.0\normalbaselineskip}
\usepackage{tabstackengine}
\setstackEOL{;}% row separator
\setstackTAB{,}% column separator
\setstacktabbedgap{1ex}% inter-column gap
\setstackgap{L}{1.5\normalbaselineskip}% inter-row baselineskip
\let\nmatrix\bracketMatrixstack  %Usage: \nmatrix{1,2,3\4,5,6}
\newcommand\cv[1]{\setstackEOL{,}\bracketMatrixstack{#1}} %usage: \cv{1,2,3}

% Custom Math Coqmmands
\newcommand{\vt}{\vskip 5mm} % vertical space
\newcommand{\fl}{\noindent\textbf} % first line
\newcommand{\Fl}{\vt\noindent\textbf} % first line with space above
\newcommand{\norm}[1]{\left\lVert#1\right\rVert} % norm
\newcommand{\pnorm}[1]{\left\lVert#1\right\rVert_p} % p-norm
\newcommand{\qnorm}[1]{\left\lVert#1\right\rVert_q} % q-norm
\newcommand{\1}[1]{\textbf{1}_{\left[#1\right]}} % indicator function 
\def\limn{\lim_{n\to\infty}} % shortcut for lim as n-> infinity
\def\sumn{\sum_{n=1}^{\infty}} % shortcut for sum from n=1 to infinity
\def\sumkn{\sum_{k=1}^{n}} % shortcut for sum from k=1 to n
\def\sumin{\sum_{i=1}^{n}} % shortcut for sum from i=1 to n
\def\SAs{\sigma\text{-algebras}} % shortcut for $\sigma$-algebras
\def\SA{\sigma\text{-algebra}} % shortcut for $\sigma$-algebra
\def\Ft{\mathcal{F}_t} % time-indexed sigma-algebra (t)
\def\Fs{\mathcal{F}_s} % time-indexed sigma-algebra (s)
\def\F{\mathcal{F}} % sigma-algebra
\def\G{\mathcal{G}} % sigma-algebra
\def\R{\mathbb{R}} % Real numbers
\def\N{\mathbb{N}} % Natural numbers
\def\Z{\mathbb{Z}} % Integers
\def\E{\mathbb{E}} % Expectation
\def\P{\mathbb{P}} % Probability
\def\Q{\mathbb{Q}} % Q probability
\def\dist{\text{dist}} %Text 'dist' for things like 'dist(x,y)'
\newcommand{\indep}{\perp \!\!\! \perp}  %independence symbol
\def\Var{\mathrm{Var}} % Variance
\def\tr{\mathrm{tr}} % trace

% Brackets and Parentheses
\def\[{\left [}
    \def\]{\right ]}
% \def\({\left (}
%   \def\){\right )}



\usepackage{color}
\definecolor{Red}{rgb}{1,0,0}
\definecolor{Blue}{rgb}{0,0,1}
\definecolor{Purple}{rgb}{.5,0,.5}
\def\red{\color{Red}}
\def\blue{\color{Blue}}
\def\gray{\color{gray}}
\def\purple{\color{Purple}}
\definecolor{RoyalBlue}{cmyk}{1, 0.50, 0, 0}
\newcommand{\dempfcolor}[1]{{\color{RoyalBlue}#1}} 
\newcommand{\demph}[1]{\dempfcolor{{\sl #1}}}

\usepackage{emoji}
% comment exactly one of the following line to show / hide the solutions
% \newcommand{\solution}[1]{{\purple #1}} % uncomment to show the solutions
\newcommand{\solution}[1]{} % uncomment to hide the solutions



\title{Math 472: Homework 01 \\Statistical Inference}
\date{Last updated: \today}
% \author{mh}

\begin{document}
\begin{center}
  \section*{Math 472: Homework 01} %
  \textit{Due Wednesday, September 10 (at the beginning of class)}
\end{center}


\begin{problem}
  Install a bunch of software on your laptop:
  \begin{enumerate}[(a)]
    \item Install the statistical software \texttt{R}, available at the \url{https://cran.rstudio.com/}
    \item Install \texttt{RStudio Desktop}, available at
    \url{https://posit.co/download/rstudio-desktop/}
    \item Do the \texttt{R} tutorial
    \item DO \url{https://swcarpentry.github.io/shell-novice/} or
    \url{https://carpentries-incubator.github.io/git-novice-branch-pr/} or \url{https://cecileane.github.io/computingtools/pages/notes0922-markdown.html}
    \item Write an R function to generate and plot a 1 dimensional brownian motion
  \end{enumerate}
\end{problem}


\begin{remark}
  The next problem is intended to introduce/review some central concepts in
  probability. For the remainder of this course, any terms defined using
  \demph{this format} are precise definitions and should be memorized (they
  are fair game for in-class quizzes) Any terms that \textbf{are bolded} are
  important and should be reviewed if you are not familiar with them.
\end{remark}

\begin{problem}
  An urn contains 5 balls, three red balls and two blue balls:
  \begin{center}
    \emoji{red-circle} \emoji{red-circle} \emoji{red-circle}
    \emoji{blue-circle} \emoji{blue-circle}
  \end{center}
  We consider the problem of sampling 3 balls from the urn, drawing the balls
  ``without replacement''. This means we draw one ball at random, then draw
  another ball at random, and then draw a third ball at random, without ever
  putting any of the balls back into the urn.

  For $k=1,2,$ and $3$, we will use the notation $R_{k}$ to denote the event
  that the $k^{\rm th}$ drawn ball is red, and $B_{k}$ to denote the event
  that the $k^{\rm th}$ drawn ball is blue. Obviously, $\P\left[R_{1}
  \right]=3/5$ and $\P\left[B_{1} \right]=2/5$.
  \begin{enumerate}[(a)]
    \item \label{item:1} Compute the conditional probabililities
    $\P\left[R_{2}\mid B_{1} \right]$ and $\P\left[R_{2}\mid R_{1} \right]$.
    \item Use the\textbf{ Law of Total Probability} (Theorem 2.8 in the textbook, p70)
    and your answer to part \ref{item:1} to compute $\P\left[R_{2} \right]$.
    \item \label{item:2} If $E$ and $F$ are events, we use the notation $EF$
    to denote the event that both $E$ and $F$ occur (i.e., $EF=E\cap F$).
    Compute the probabilities of the four events $R_{1}R_{2}$, $R_{1}B_{2}$,
    $B_{1}R_{2}$ and $B_{1}B_{2}$.
    \item Use the Law of Total Probability and your answer to part
    \ref{item:2} to compute $\P\left[R_{3} \right]$.
    \item In the remainder of this problem, we will compute the expected
    proportion of red balls among our 3 draws. To do this, we will introduce a
    standard technique: the use of indicator functions.

    Given an event $E$, the \demph{indicator function of $E$} is the function
    \begin{equation*}
      \mathbf{1}_{E} = \left\{ \begin{array}{l@{\quad:\quad}l} 1 & \text{the event }E
          \text{ occurs}\\ 0& \text{the event }E \text{ does not occur} \end{array}\right.
    \end{equation*}
    Indicator functions are random variables. Taking $E=R_{k}$, we have
    \begin{equation*}
      \mathbf{1}_{R_{k}} = \left\{ \begin{array}{l@{\quad:\quad}l} 1 & \text{the }k^{\rm th}
          \text{ ball drawn is red}\\ 0& \text{the }k^{\rm th} \text{ ball drawn is blue.} \end{array}\right.
    \end{equation*}
    A random variable is \demph{discrete} if it can assume only a finite or
    countably infinite number of distinct values.

    % If $X$ is a random variable, the \demph{support} of $X$ is the smallest
    % closed set $S$ such that $\P\left[X\in S \right]=1$. We will denote the
    % support of $X$ either by $\text{supp}(X)$ or more commonly $S_{X}$.
    % Informally, the support of $X$ is the set of all possible values that it
    % can take.

    % A random variable $X$ is \demph{discrete} if $S_{X}$ is
    % countable.

    If $X$ is a discrete random variable, and $S_{X}\subseteq \R$ is the set
    of possible values that $X$ can take, then the \demph{expectation} of $X$,
    denoted $\E\left[X \right]$, is defined as
    \begin{equation*}
      \E\left[X \right]:= \sum_{x\in S_{X}}x \P\left[X=x \right],
    \end{equation*}
    provided that this sum converges absolutely.

    
    Using the above definition of expectation, prove that
    $\E\left[\mathbf{1}_{R_{k}} \right]= \P\left[R_{k} \right]$ for $k=1,2,3$.
    \item Observe that
    \begin{equation*}
      (\#\text{ of red balls in 3 draws}) = \mathbf{1}_{R_{1}}+\mathbf{1}_{R_{2}}+\mathbf{1}_{R_{3}},
    \end{equation*}
    and hence
    \begin{equation}\label{eq:1}
      \text{(the proportion of red balls in 3 draws)} = \frac{\mathbf{1}_{R_{1}}+\mathbf{1}_{R_{2}}+\mathbf{1}_{R_{3}}}{3}
    \end{equation}
    The \textbf{linearity of expectation} says that if $X,Y$ are random
    variables, and $a,b$ are scalars, then
    $\E\left[X+Y \right]= \E\left[X \right]+ \E\left[Y \right]$, and
    $\E\left[aX+b \right]=a \E\left[X \right]+b$. Use the linearity of
    expectations and \cref{eq:1} to compute the expected proportion of red
    balls in 3 draws. \textit{(Note: if you've done all parts of this problem
      correct, you'll get 3/5.)}
    
  \end{enumerate}
\end{problem}

\end{document}