\documentclass[10pt]{article}
% Math Packages
\usepackage{amsmath, mathtools}
\usepackage{amssymb}
\usepackage{amsthm}
\usepackage{amsfonts}
\usepackage{bbm}
\usepackage{breqn}
\usepackage[margin=1in]{geometry}
\usepackage{graphicx}
\usepackage{tikz}
\usetikzlibrary{arrows.meta}
\usetikzlibrary{calc}
\usepackage{forest}
\usepackage{tikz-qtree}
\graphicspath{ {./images/} }
\usepackage{hyperref}
\usepackage[capitalize]{cleveref}
\usepackage[shortlabels]{enumitem}
\usetikzlibrary{arrows,matrix,positioning}
\usepackage{multicol}

\usepackage{listings}
\usepackage{xcolor}

\crefname{problem}{Problem}{Problems}
\Crefname{problem}{Problem}{Problems}

\lstset{
  language=R,
  basicstyle=\ttfamily\footnotesize,
  keywordstyle=\color{blue},
  commentstyle=\color{gray},
  stringstyle=\color{red!70!black},
  framesep=2pt,
  framexleftmargin=0pt, 
  frame=single,
  breaklines=false,
  showstringspaces=false
}
% for the pipe symbol
\usepackage[T1]{fontenc}

% Citing theorems by name. (source: https://tex.stackexchange.com/questions/109843/cleveref-and-named-theorems)
\makeatletter
\newcommand{\ncref}[1]{\cref{#1}\mynameref{#1}{\csname r@#1\endcsname}}

\def\mynameref#1#2{%
  \begingroup
  \edef\@mytxt{#2}%
  \edef\@mytst{\expandafter\@thirdoffive\@mytxt}%
  \ifx\@mytst\empty\else
  \space(\nameref{#1})\fi
  \endgroup
}
\makeatother

% Colorful Notes
\usepackage{color} \definecolor{Red}{rgb}{1,0,0} \definecolor{Blue}{rgb}{0,0,1}
\definecolor{Purple}{rgb}{.5,0,.5} \def\red{\color{Red}} \def\blue{\color{Blue}}
\def\gray{\color{gray}} \def\purple{\color{Purple}}
\newcommand{\rnote}[1]{{\red [#1]}} % \rnote{foo} gives '[foo]' in red
\newcommand{\pnote}[1]{{\purple [#1]}} % \pnote{foo} gives '[foo]' in purple
\newcommand{\bnote}[1]{{\blue #1}} % \bnote{foo} gives 'foo' in blue
\newcommand{\gnote}[1]{{\gray #1}} % \gnote{foo} gives 'foo' in gray
\newcommand{\Max}[1]{{\purple [#1]}} % \bnote{foo} then 'foo' is blue


% Claim numbering (the counter restarts after each proof environment)
\newcounter{claimcount}
\setcounter{claimcount}{0}
\newenvironment{claim}{\refstepcounter{claimcount}\par\addvspace{\medskipamount}\noindent\textbf{Claim \arabic{claimcount}:}}{}
\usepackage{etoolbox}
\AtBeginEnvironment{proof}{\setcounter{claimcount}{0}}
\newenvironment{claimproof}{\par\addvspace{\medskipamount}\noindent\textit{Proof of Claim  \arabic{claimcount}.}}{\hfill\ensuremath{\qedsymbol} \tiny{Claim}

  \medskip}
% Add claim support to cleverref
\crefname{claimcount}{Claim}{Claims}


% Math Environments
\newtheorem{theorem}{Theorem}
\newtheorem{assumption}[theorem]{Assumption}
\newtheorem{lemma}[theorem]{Lemma}
\newtheorem{proposition}[theorem]{Proposition}
\newtheorem{corollary}[theorem]{Corollary}
\newtheorem{question}[theorem]{Question}
\theoremstyle{definition}
\newtheorem*{definition}{Definition}
\newtheorem{remark}[theorem]{Remark}
\newtheorem{example}[theorem]{Example}
\newtheorem{notation}[theorem]{Notation}
\newtheorem{problem}[theorem]{Problem}

% Matrices and Column Vectors. 
\usepackage{stackengine}
\setstackgap{L}{1.0\normalbaselineskip}
\usepackage{tabstackengine}
\setstackEOL{;}% row separator
\setstackTAB{,}% column separator
\setstacktabbedgap{1ex}% inter-column gap
\setstackgap{L}{1.5\normalbaselineskip}% inter-row baselineskip
\let\nmatrix\bracketMatrixstack  %Usage: \nmatrix{1,2,3\4,5,6}
\newcommand\cv[1]{\setstackEOL{,}\bracketMatrixstack{#1}} %usage: \cv{1,2,3}

% Custom Math Coqmmands
\newcommand{\vt}{\vskip 5mm} % vertical space
\newcommand{\fl}{\noindent\textbf} % first line
\newcommand{\Fl}{\vt\noindent\textbf} % first line with space above
\newcommand{\norm}[1]{\left\lVert#1\right\rVert} % norm
\newcommand{\pnorm}[1]{\left\lVert#1\right\rVert_p} % p-norm
\newcommand{\qnorm}[1]{\left\lVert#1\right\rVert_q} % q-norm
\newcommand{\1}[1]{\textbf{1}_{\left[#1\right]}} % indicator function 
\def\limn{\lim_{n\to\infty}} % shortcut for lim as n-> infinity
\def\sumn{\sum_{n=1}^{\infty}} % shortcut for sum from n=1 to infinity
\def\sumkn{\sum_{k=1}^{n}} % shortcut for sum from k=1 to n
\def\sumin{\sum_{i=1}^{n}} % shortcut for sum from i=1 to n
\def\SAs{\sigma\text{-algebras}} % shortcut for $\sigma$-algebras
\def\SA{\sigma\text{-algebra}} % shortcut for $\sigma$-algebra
\def\Ft{\mathcal{F}_t} % time-indexed sigma-algebra (t)
\def\Fs{\mathcal{F}_s} % time-indexed sigma-algebra (s)
\def\F{\mathcal{F}} % sigma-algebra
\def\G{\mathcal{G}} % sigma-algebra
\def\R{\mathbb{R}} % Real numbers
\def\N{\mathbb{N}} % Natural numbers
\def\Z{\mathbb{Z}} % Integers
\def\E{\mathbb{E}} % Expectation
\def\P{\mathbb{P}} % Probability
\def\Q{\mathbb{Q}} % Q probability
\def\dist{\text{dist}} %Text 'dist' for things like 'dist(x,y)'
\newcommand{\indep}{\perp \!\!\! \perp}  %independence symbol
\def\Var{\mathrm{Var}} % Variance
\def\tr{\mathrm{tr}} % trace

% Brackets and Parentheses
\def\[{\left [}
    \def\]{\right ]}
% \def\({\left (}
%   \def\){\right )}



\usepackage{color}
\definecolor{Red}{rgb}{1,0,0}
\definecolor{Blue}{rgb}{0,0,1}
\definecolor{Purple}{rgb}{.5,0,.5}
\def\red{\color{Red}}
\def\blue{\color{Blue}}
\def\gray{\color{gray}}
\def\purple{\color{Purple}}
\definecolor{RoyalBlue}{cmyk}{1, 0.50, 0, 0}
\newcommand{\dempfcolor}[1]{{\color{RoyalBlue}#1}} 
\newcommand{\demph}[1]{\textcolor{RoyalBlue}{\textbf{\slshape #1}}} % Slanted

\usepackage{emoji}
% comment exactly one of the following line to show / hide the solutions
% \newcommand{\solution}[1]{{\purple #1}} % uncomment to show the solutions
\newcommand{\solution}[1]{} % uncomment to hide the solutions




\begin{document}
\begin{center}
  \section*{Math 472: Homework 05}
  \textit{Due Friday, February 27}
\end{center}


\begin{problem}[Exercise 8.31]
  In a study to compare the perceived effects of two pain relievers, 200 randomly selected
adults were given the first pain reliever, and 93\% indicated appreciable pain relief. Of the 450
individuals given the other pain reliever, 96\% indicated experiencing
appreciable relief.
\begin{enumerate}[(a)]
  \item Give an estimate for the difference in the proportions of all adults who would indicate
  perceived pain relief after taking the two pain relievers. Provide a bound on the error of
  estimation.
  \item Based on your answer to part (a), is there evidence that proportions experiencing relief
  differ for those who take the two pain relievers? Why?
\end{enumerate}
\end{problem}

\begin{problem}[Exercise 8.17]
  If $Y$ has a binomial distribution with parameter $n$ and $p$, the
  $\widehat{p}=\frac{Y}{n}$ is an unbiased estimator of $p$. Another estimator
  is $\widehat{q}=\frac{Y+1}{n+2}$.
  \begin{enumerate}[(a)]
    \item What is the bias of $\widehat{q}$?
    \item Derive $\mathtt{MSE}(\widehat{p})$ and $\text{MSE}(\widehat{q})$.
    \item For what values of $p$ is $\text{MSE}(\widehat{p})< \text{MSE}(\widehat{q})$?
  \end{enumerate}
\end{problem}

\begin{problem}[Exercise 8.39]
  Suppose the random variable $Y$ has a gamma distribution with shape parameter
  $\alpha=2$ and unknown scale parameter $\beta$ (see section 4.6 in
  textbook). It can be shown that $2Y/\beta$ has a chi-squared distribution with 4
  degrees of freedom (you don't have to show this).

  \begin{enumerate}[(a)]
    \item Using $2Y/\beta$ as a pivotal quantity, derive a $90\%$ confidence interval
    of $\beta$.
    \item Using \textrm{R}, test your confidence interval by generating 10,000
    samples $Y_{1},\ldots,Y_{10000}$ (each with a randomly chosen numerical
    value for $\beta$) and counting the proportion of times that your
    confidence interval contains the true value of $\beta$. (It should be
    close to 90\%).
  \end{enumerate}
\end{problem}

\begin{problem}[Exercise 8.40]
  Suppose that the random variable $Y$ is an observation from a normal
  distribution with unknown mean $\mu$ and variance $1$.
  \begin{enumerate}[(a)]
    \item Find a 95\% confidence interval for $\mu$. 
    \item Find a 95\% upper confidence interval for $\mu$. 
    \item Find a 95\% lower confidence interval for $\mu$. 
  \end{enumerate}
\end{problem}

\begin{problem}[Excercise 8.41]
  Suppose that $Y$ is normally distributed with mean $\mu=0$ and unknown
  variance $\sigma^{2}$. Then $Y^{2}/\sigma^{2}$ has a chi-squared
  distribution with 1 degree of freedom. Use the pivotal quantity
  $Y^{2}/\sigma^{2}$ to find
  \begin{enumerate}[(a)]
    \item A $95\%$ confidence interval for $\sigma^{2}$.
    \item A $95\%$ upper confidence interval for $\sigma^{2}$.
    \item A $95\%$ lower confidence interval for $\sigma^{2}$.
  \end{enumerate}
\end{problem}


\begin{problem}[Part of exercise 6.18]
  The Pareto distribution is a ``power-law'' distribution used for modeling
  things like income distributions and earthquake magnitudes. Specifically,
  given a random variable $X$, we say that distribution of $X$ is a member of
  the \demph{Pareto family of distributions} if its distribution function is
  \begin{equation}\label{eq:2}
    F_{X}(x) = \P\left[X \leq x \right] 
    = \left\{ 
      \begin{array}{l@{\quad:\quad}l}
        1-\left( \frac{\beta}{x} \right)^{\alpha} & x \geq \beta\\
        0 & x<\beta
      \end{array}\right.
  \end{equation}
  for some positive parameters $\alpha$ and $\beta$.
  \begin{enumerate}[(a)]
    \item Big-O notation: Let $f$ and $g$ be real-valued functions whose
    domains include the positive real numbers. We write $f(x) =O(g(x))$ as
    $x \to \infty$ if there exist constants $M,k>0$ such that
    \begin{equation*}
      \frac{|f(x)|}{|g(x)|} \leq M
    \end{equation*}
    for all $x>k$. (This is read as ``$f$ is big-O of $g$''.)

    Show that $\P\left[X \geq x\right] = O(\frac{1}{x^{\alpha}})$ as
    $x \to \infty$. (In words, this says that the upper tail of the
    probability distribution of $X$ decays like $\frac{1}{x^{\alpha}}$.)
    
    \item Derive the density function of $X$. Make sure to specify the
    function on all parts of its domain: both when $x \geq \beta$ and when
    $x<\beta$.
    \item Show that the improper integral
    $\int_{1}^{\infty}\frac{1}{x^{\delta}}dx$ diverges to $+\infty$ if
    $\delta \leq 1$, but converges to $\frac{1}{\delta-1}$ if $\delta>1$.
    \item What is $\E\left[X \right]$? Consider the cases $\alpha \leq 1$ and
    $\alpha > 1$ separately.
  \end{enumerate}

\end{problem}


\begin{problem}[Part of exercise 9.28]
  Let $X_{1},\ldots,X_{n}$ be a random sample from the distribution $F_{X}$ (defined in \cref{eq:2}), and let
  $T$ be the statistic $T(X_{1},\ldots,X_{n})=\min(X_{1},\ldots,X_{n})$.
  \begin{enumerate}[(a)]
    \item  Show that the sampling distribution of $T$ is
    \begin{equation}\label{eq:1}
      F_{T}(t) = \left\{ \begin{array}{l@{\quad:\quad}l} 
          1- \left( \frac{\beta}{t} \right)^{\alpha n} & t \geq \beta \\
          0 &t < \beta \end{array}\right..
    \end{equation}
    (\textit{Hint:} For every real number $t$, we have
    $\left[ \min(X_{1},X_{2},\ldots,X_{n}) > t\right] = \left[ X_{1}> t,
      X_{2}> t,\ldots, X_{n}> t\right]$).
    \item Derive the density function of $T$ from \cref{eq:1}. Make sure to
    specify the function on all parts of its domain: both when $t \geq \beta$
    and when $t<\beta$.
    \item Qualitatively describe what happens to the graph of $f_{T}$ as $n$
    becomes very large. Sketch this situation, and interpret your picture: is
    $T$ likely to be close to $\beta$ or far from $\beta$ when $n$ is large?
  \end{enumerate}
\end{problem}

\begin{problem}[Exercise 8.15]
  Let $X_{1},\ldots,X_{n}$ be a random sample from the Pareto distribution
  with probability density function
  \begin{equation*}
    f_{X}(x) = \left\{ \begin{array}{l@{\quad:\quad}l} 3\beta^{3}x^{-4} &  x \geq \beta
        \\ 0& x<\beta\end{array}\right.,
  \end{equation*}
  where $\beta>0$ is unknown. (So $X$ is a Pareto distribution with
  $\alpha=3$). We shall consider the estimator
  $\widehat{\beta}=T(X_{1},\ldots,X_{n})=\min(X_{1},\ldots,X_{n})$.
  \begin{enumerate}[(a)]
    \item What is the bias of $\widehat{\beta}$?
    \item What is $\text{MSE}(\widehat{\beta})$?
  \end{enumerate}
\end{problem}



% A \demph{statistical model} is a collection of probability
% distributions which are all defined on a common sample space; that is, a
% collection
% \begin{equation*}
%   \mathcal{P} = \left\{F_{\theta}:\theta\in \Theta\right\},
% \end{equation*}
% where $\Theta$ is an arbitary index set called the \demph{parameter space}
% and for each $\theta\in \Theta$, $F_{\theta}$ is a probability
% distribution on the sample space. Moreover, if the parameter space
% $\Theta$ is a subset of $\R^{d}$ for some $d \geq 1$, then we say that the
% statistical model $\mathcal{P}$ is \demph{parametric}.


% Note the Pareto family is an example of a parametric statistical model
% with two parameters $\alpha$ and $\beta$, i.e.,
% $\Theta=\left\{(\alpha,\beta)\in \R^{2}: \alpha,\beta>0\right\}$.
\end{document}