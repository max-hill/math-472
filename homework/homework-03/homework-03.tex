\documentclass[10pt]{article}
% Math Packages
\usepackage{amsmath, mathtools}
\usepackage{amssymb}
\usepackage{amsthm}
\usepackage{amsfonts}
\usepackage{bbm}
\usepackage{breqn}
\usepackage[margin=1in]{geometry}
\usepackage{graphicx}
\usepackage{tikz}
\usetikzlibrary{arrows.meta}
\usetikzlibrary{calc}
\usepackage{forest}
\usepackage{tikz-qtree}
\graphicspath{ {./images/} }
\usepackage{hyperref}
\usepackage[capitalize]{cleveref}
\usepackage[shortlabels]{enumitem}
\usetikzlibrary{arrows,matrix,positioning}
\usepackage{multicol}

\usepackage{listings}
\usepackage{xcolor}

\lstset{
  language=R,
  basicstyle=\ttfamily\footnotesize,
  keywordstyle=\color{blue},
  commentstyle=\color{gray},
  stringstyle=\color{red!70!black},
  framesep=2pt,
  framexleftmargin=0pt, 
  frame=single,
  breaklines=false,
  showstringspaces=false
}
% for the pipe symbol
\usepackage[T1]{fontenc}

% Citing theorems by name. (source: https://tex.stackexchange.com/questions/109843/cleveref-and-named-theorems)
\makeatletter
\newcommand{\ncref}[1]{\cref{#1}\mynameref{#1}{\csname r@#1\endcsname}}

\def\mynameref#1#2{%
  \begingroup
  \edef\@mytxt{#2}%
  \edef\@mytst{\expandafter\@thirdoffive\@mytxt}%
  \ifx\@mytst\empty\else
  \space(\nameref{#1})\fi
  \endgroup
}
\makeatother

% Colorful Notes
\usepackage{color} \definecolor{Red}{rgb}{1,0,0} \definecolor{Blue}{rgb}{0,0,1}
\definecolor{Purple}{rgb}{.5,0,.5} \def\red{\color{Red}} \def\blue{\color{Blue}}
\def\gray{\color{gray}} \def\purple{\color{Purple}}
\newcommand{\rnote}[1]{{\red [#1]}} % \rnote{foo} gives '[foo]' in red
\newcommand{\pnote}[1]{{\purple [#1]}} % \pnote{foo} gives '[foo]' in purple
\newcommand{\bnote}[1]{{\blue #1}} % \bnote{foo} gives 'foo' in blue
\newcommand{\gnote}[1]{{\gray #1}} % \gnote{foo} gives 'foo' in gray
\newcommand{\Max}[1]{{\purple [#1]}} % \bnote{foo} then 'foo' is blue


% Claim numbering (the counter restarts after each proof environment)
\newcounter{claimcount}
\setcounter{claimcount}{0}
\newenvironment{claim}{\refstepcounter{claimcount}\par\addvspace{\medskipamount}\noindent\textbf{Claim \arabic{claimcount}:}}{}
\usepackage{etoolbox}
\AtBeginEnvironment{proof}{\setcounter{claimcount}{0}}
\newenvironment{claimproof}{\par\addvspace{\medskipamount}\noindent\textit{Proof of Claim  \arabic{claimcount}.}}{\hfill\ensuremath{\qedsymbol} \tiny{Claim}

  \medskip}
% Add claim support to cleverref
\crefname{claimcount}{Claim}{Claims}


% Math Environments
\newtheorem{theorem}{Theorem}
\newtheorem{assumption}[theorem]{Assumption}
\newtheorem{lemma}[theorem]{Lemma}
\newtheorem{proposition}[theorem]{Proposition}
\newtheorem{corollary}[theorem]{Corollary}
\newtheorem{question}[theorem]{Question}
\theoremstyle{definition}
\newtheorem*{definition}{Definition}
\newtheorem{remark}[theorem]{Remark}
\newtheorem{example}[theorem]{Example}
\newtheorem{notation}[theorem]{Notation}
\newtheorem{problem}[theorem]{Problem}

% Matrices and Column Vectors. 
\usepackage{stackengine}
\setstackgap{L}{1.0\normalbaselineskip}
\usepackage{tabstackengine}
\setstackEOL{;}% row separator
\setstackTAB{,}% column separator
\setstacktabbedgap{1ex}% inter-column gap
\setstackgap{L}{1.5\normalbaselineskip}% inter-row baselineskip
\let\nmatrix\bracketMatrixstack  %Usage: \nmatrix{1,2,3\4,5,6}
\newcommand\cv[1]{\setstackEOL{,}\bracketMatrixstack{#1}} %usage: \cv{1,2,3}

% Custom Math Coqmmands
\newcommand{\vt}{\vskip 5mm} % vertical space
\newcommand{\fl}{\noindent\textbf} % first line
\newcommand{\Fl}{\vt\noindent\textbf} % first line with space above
\newcommand{\norm}[1]{\left\lVert#1\right\rVert} % norm
\newcommand{\pnorm}[1]{\left\lVert#1\right\rVert_p} % p-norm
\newcommand{\qnorm}[1]{\left\lVert#1\right\rVert_q} % q-norm
\newcommand{\1}[1]{\textbf{1}_{\left[#1\right]}} % indicator function 
\def\limn{\lim_{n\to\infty}} % shortcut for lim as n-> infinity
\def\sumn{\sum_{n=1}^{\infty}} % shortcut for sum from n=1 to infinity
\def\sumkn{\sum_{k=1}^{n}} % shortcut for sum from k=1 to n
\def\sumin{\sum_{i=1}^{n}} % shortcut for sum from i=1 to n
\def\SAs{\sigma\text{-algebras}} % shortcut for $\sigma$-algebras
\def\SA{\sigma\text{-algebra}} % shortcut for $\sigma$-algebra
\def\Ft{\mathcal{F}_t} % time-indexed sigma-algebra (t)
\def\Fs{\mathcal{F}_s} % time-indexed sigma-algebra (s)
\def\F{\mathcal{F}} % sigma-algebra
\def\G{\mathcal{G}} % sigma-algebra
\def\R{\mathbb{R}} % Real numbers
\def\N{\mathbb{N}} % Natural numbers
\def\Z{\mathbb{Z}} % Integers
\def\E{\mathbb{E}} % Expectation
\def\P{\mathbb{P}} % Probability
\def\Q{\mathbb{Q}} % Q probability
\def\dist{\text{dist}} %Text 'dist' for things like 'dist(x,y)'
\newcommand{\indep}{\perp \!\!\! \perp}  %independence symbol
\def\Var{\mathrm{Var}} % Variance
\def\tr{\mathrm{tr}} % trace

% Brackets and Parentheses
\def\[{\left [}
    \def\]{\right ]}
% \def\({\left (}
%   \def\){\right )}



\usepackage{color}
\definecolor{Red}{rgb}{1,0,0}
\definecolor{Blue}{rgb}{0,0,1}
\definecolor{Purple}{rgb}{.5,0,.5}
\def\red{\color{Red}}
\def\blue{\color{Blue}}
\def\gray{\color{gray}}
\def\purple{\color{Purple}}
\definecolor{RoyalBlue}{cmyk}{1, 0.50, 0, 0}
\newcommand{\dempfcolor}[1]{{\color{RoyalBlue}#1}} 
\newcommand{\demph}[1]{\textcolor{RoyalBlue}{\textbf{\slshape #1}}} % Slanted

\usepackage{emoji}
% comment exactly one of the following line to show / hide the solutions
% \newcommand{\solution}[1]{{\purple #1}} % uncomment to show the solutions
\newcommand{\solution}[1]{} % uncomment to hide the solutions




\begin{document}
\begin{center}
  \section*{Math 472: Homework 03} %
  \textit{Due Monday Feb 9}
\end{center}

\begin{problem}[Binomial distribution, revisited]
  Let $X\sim \text{Bin(n,p)}$. (That is, $X$ is a binomial random variable
  with number of trials $n$ and success probability $p$). One way to
  interpret the random variable $X$ is as follows
  \begin{equation*}
    X = \text{(the number of heads in $n$ coin flips)},
  \end{equation*}
  where $p$ is the probability that the coin lands on heads. Or, more
  formally,
  \begin{equation*}
    X = \xi_{1}+\xi_{2}+\ldots+\xi_{n}
  \end{equation*}
  where each $\xi_{1},\ldots\xi_{n}$ are independent with
  \begin{equation*}
    \xi_{i}= \left\{ \begin{array}{l@{\quad:\quad}l} 1 &
        \text{with probability }p\\ 0 & \text{with probability }1-p \end{array}\right.
  \end{equation*}
  \begin{enumerate}[(a)]
    \item Use this characterization to compute the expected value and variance
    of $X$.
    \item Use \texttt{R} to draw a random sample of size 50 from the
    distribution of $X$ (pick any value of $p\in (0,1)$ that you want) and
    compute the sample mean and sample variance. Compare with the values you
    got in part (a).
  \end{enumerate}
\end{problem}

\begin{problem}[\texttt{R} problem]
  \begin{enumerate}[(a)]
    \item []
    \item Let $Z_{1},\ldots,Z_{6}$ be a sample of standard normal random variables.
    Use \texttt{R} to find
    $\P\left[ \sum_{i=1}^{6}Z_{i}^{2}\leq 6\right]$.
    \item Let $Y_{1},\ldots,Y_{10} $ be a sample of normal random variables
    with mean $\mu$ and variance $\sigma^{2}=1$. Let $S^{2}$ be the sample
    variance. Use \texttt{R} to find $\P\left[S^{2}\geq 3\right]$.
  \end{enumerate}
\end{problem}



\begin{problem}
  \label{problem:1}
  Refer to example 7.2 in the textbook (i.e., examples 57 in the typed lecture
  notes). The amount of fill dispensed by a bottling machine is normally
  distributed with $\sigma=1$. If $n = 9$ bottles are randomly selected from
  the output of the machine, we found that the probability that the sample
  mean will be within .3 ounce of the true mean is .6318. Suppose that $Y$ is
  to be computed using a sample of size $n$.
  \begin{enumerate}[(a)]
    \item If $n=16$, what is $\P\left[\left| \overline{Y} -\mu \right| \leq
      .3\right]$?
    \item When $\P\left[\left| \overline{Y}-\mu \right| \leq .3\right]$ when
    $\overline{Y}$ is computed using samples of sizes $n=25,36,49$ and $64$.
    \item What patterns do you observe among the values for $\P\left[\left|
        \overline{Y}-\mu \right| \leq .3\right]$ for the various values of
    $n$?
    \item Do the results you obtained in part (b) appear to be consistent with
    the result obtained in Example 7.3 in the textbook?
  \end{enumerate}
\end{problem}

\begin{problem}
  Refer to example 7.2 in the textbook. Assume now that the amount of fill
  dispensed by the bottling machine is normally distributed with $\sigma=2$
  ounces.
  \begin{enumerate}[(a)]
    \item If $n=9$ bottles are randomly selected, what is $\P\left[\left|
        \overline{Y}-\mu \right| \leq .3\right]$? Compare this with the answer
    found in Example 7.2
    \item When $\P\left[\left| \overline{Y}-\mu \right| \leq .3\right]$ when
    $\overline{Y}$ is computed using samples of sizes $n=25,36,49$ and $64$.
    \item What patterns do you observe among the values for $\P\left[\left|
        \overline{Y}-\mu \right| \leq .3\right]$ for the various values of
    $n$?
    \item How do the respective probabilities obtained in this problem (where
    $\sigma=2$) compare to those obtained in \cref{problem:1} (where $\sigma=1$)?
  \end{enumerate}
\end{problem}


\begin{problem}
  \label{problem:2}
  A forester studying the effects of fertilization on certain pine forests in
  the Southeast is interested in estimating the average basal area of pine
  trees. In studying basal areas of similar trees for many years, he has
  discovered that these measurements (in square inches) are normally
  distributed with standard deviation approximately $\sigma=4$ square inches. If the
  forester samples $n = 9$ trees, find the probability that the sample mean will
  be within 2 square inches of the population mean.
\end{problem}

\begin{problem}
  Suppose the forester in \cref{problem:2} would like the sample mean to be
  within 1 square inch of the population mean, with probability .90. How many
  trees must he measure in order to ensure this degree of accuracy?
\end{problem}


\begin{problem}[Another \texttt{R} problem]
  \begin{enumerate}[(a)]
    \item []
    \item Use \texttt{R} to plot histograms comparing the standard normal
    distribution with t-distribution, for various degrees of freedom (e.g., 5,
    10, etc). [The point of this problem is for you to visually observe how the
    t-distribution converges to the standard normal as the degrees of freedom
    increase.] The following code may be helpful:

\begin{verbatim}
    ## Plotting a histogram of a t-distribution
    d = 10 # degrees of freedom for the t-distribution (play around with this)

    # generate a ton of samples from t and z distributions
    t_samples = rt(n=100000, df=d)
    z_samples = rnorm(n=100000,mean=0,sd=1)

    # truncate the datasets to only samples within the range (-6,6). This is
    # needed to make the plot pretty
    z_truncated<- z_samples[z_samples >= -6 & z_samples <= 6]
    t_truncated<- t_samples[t_samples >= -6 & t_samples <= 6]

    # set histogram box widths to 0.1, and have them range from -6 to 6.
    my_breaks<-seq(-6,6, by=.1)

    # plot the histograms
    hist(z_truncated, probability=TRUE, col=rgb(1,0,0,0.4), breaks=my_breaks,
         border="red", main="t-distribution vs standard normal distribution")

    hist(t_truncated, probability=TRUE, col=rgb(0,0,1,0.4),
         breaks=my_breaks, border="blue",add=TRUE)

    legend("topright",
           legend = c("approx. distribution of Z",
                      paste("approx. t-distribution with df=",degrees_of_freedom)),
           fill=c(rgb(1,0,0,0.4), rgb(0,0,1,0.4)))
\end{verbatim}
    
    \item Suppose $T$ is a random variable with $t$-distribution and 5 degrees of freedom. The
    \texttt{R} code \texttt{pt(t,df=5)} returns $\P\left[T \leq t\right]$. Use
    this to find the probability that $T$ is greater than 2.
    \item Use \texttt{R} to find the probability that $T$ is less than $-2$.
    \item \label{item:1} Find the probability that $T$ is between $-2$ and $2$.
    \item Your answer to part \ref{item:1} is considerably less than
    $.9544=\P\left[-2 \leq Z \leq 2\right]$. Provide 1-2 sentence explanation
    of why.
  \end{enumerate}
\end{problem}

\newpage


The remaining problems involve the idea of \textit{quantiles}, defined here:

\begin{definition}[$p^{\rm th}$-quantile]
  Let $Y$ be a continuous random variable, and fix a real number $p\in (0,1)$.
  Then the \demph{$p^{\rm th}$ quantile} of $Y$, denoted $\phi_{p}$, is the
  smallest value such that
  \begin{equation*}
    \P\left[Y \leq \phi_{p}\right]  = p,
  \end{equation*}
  
  For example, when $p=1/2$, then $\phi_{\frac{1}{2}}$ is the median.
  Quantiles are also referred to using the ``percentile'' terminology, e.g.,
  if $p=.98,$ then $\phi_{.98}$ is the $98^{\rm th}$ percentile (i.e., the
  cutoff below which lies 98\% of the population). When reasoning about
  quantiles, I find it very helpful to draw pictures of bell curves with
  shaded areas.

  % When the cdf $F$ is invertible, the quantile function defined by $Q(p) :=
% \phi_{p}$ is the function inverse of $F$ (i.e., for all $Q(F(x))=x$ for all
% $x\in \R$ and $F(Q(p))=p$ for all $p\in (0,1)$).
\end{definition}

\begin{problem}
  Suppose $T$ is a random variable with $t$-distribution and 5 degrees of freedom.
  \begin{enumerate}[(a)]
    \item For any $p\in (0,1)$, the \texttt{R} code \texttt{qt(p,df=5)} gives $\phi_{p}$ (the
    $p^{\rm th}$ quantile of $T$). Let $t_{.10}$ be the number such that
    $\P\left[T>t_{.10} \right]=.10$. Find the value of $t_{.10}$.
    \item What quantile does $t_{.10}$ correspond to? What percentile?
    \item \label{item:2}  Find the value of $t_{.10}$ for t-distributions with $df=30,60$ and
    $120$.
    \item When $Z$ has a standard normal distribution, $\P\left[Z>1.282
    \right]=.10$, and $z_{.10}=1.282$. What property of the $t$ distribution
    (when compared to the standard normal dsitribution) explains the fact that
    all the values obtained in \ref{item:2} are larger than $z_{.10}=1.282$?
    \item Guess what $t_{.10}$ converges to as the number of degrees of
    freedom gets large.
  \end{enumerate}
\end{problem}

\begin{problem}[Exercise 4.12 in textbook]
  The length of time to failure (in hundreds of hours) for a transistor is a
  random variable $Y$ with cumulative distribution function given by
  \begin{equation*}
    F(y) = \left\{ \begin{array}{l@{\quad:\quad}l} 1-e^{-y^{2}} & y \geq 0\\ 0& y<0\end{array}\right.
  \end{equation*}
  \begin{enumerate}[(a)]
    \item Show that $F$ has the properties of a distribution function (i.e.,
    that $F(-\infty)=0$, $F(\infty)=1$, and $F$ is nondecreasing).
    \item Find the $.3$-quantile, $\phi_{.3}$ of $Y$. 
    \item Find the probability density function of $Y$.
    \item Find the probability that the transistor operates for at least $200$
    hours (remember the units are in hundreds of hours. You don't need to
    evaluate an integral).
    \item Find $\P\left[Y>100\mid Y \leq 200\right]$. (Again, no need to
    evaluate an integral :) ).
  \end{enumerate}
\end{problem}


\begin{problem}[Refer to \cref{problem:2}] Suppose that in the forest
  fertilization problem the population, standard deviation of basal areas is
  not known and must be estimated from the sample. If a random sample of $n=9$
  basal areas is to be measured, find a statistic $A$ such that
  \begin{equation*}
    \P\left[ \left| \overline{Y}-\mu \right| \leq A \right] = .90
  \end{equation*}
  Assume that neither the mean $\mu$ nor the variance $\sigma^{2}$ of the
  population are known.
\end{problem}

% \begin{problem}[Geometric distribution]
%   Consider a sequence of independent coin flips, each of which has probability
%   $p$ of being heads. Define a random variable $X$ as the length of the run
%   (either heads or tails) started by the first trial. (For example, if
%   \texttt{TTTH} or \texttt{HHHT} is observed).
%   \begin{enumerate}[(a)]
%     \item What is the sample space? What are the sample points? 
%     \item Find the distribution of $X$ (i.e., compute the probability mass
%     function of $X$).
%     \item Find $\E\left[X \right]$.
%     \item Use \texttt{R} to generate 50 samples from the distribution of $X$
%     (use whatever value of $p\in (0,1)$ you want). What is the sample mean?
%     What is the sample variance?
%   \end{enumerate}
% \end{problem}

% \begin{problem}%[CLT: this problem is similar to problem 9 in homework 9]
%   A mandolin-making machine in Mordecai's mandolin manufactory makes about 5\%
%   defective mandolins even when properly functioning. The mandolins are then
%   packed in crates containing 1900 mandolins each. A crate is examined and
%   found to contain 115 defective mandolins. What is the approximate
%   probability of finding at least this many defective mandolins if the machine
%   is properly adjusted? If you were Mordecai, would hire a technician to check
%   out the machine?

%   \textit{Hint:} The following may be useful $\sqrt{1900}=10 \sqrt{19}$.
% \end{problem}



% \begin{problem}%[CLT] % similar to hw 9, problem 6
%   The nation of Oceania has always been at war with Eurasia. To establish
%   peace, Oceania's leadership has unanimously authorized a preemptive strike
%   against 48 pre-selected military targets in Eurasian territory. Analysts at
%   the \textit{Ministry of Peace} estimate that each target has a probability
%   of $3/4$ of being destroyed in this first strike.

%   The operation will be regarded as a strategic success under Oceania's peace
%   doctrine if at least 30 targets are destroyed. Using the Central Limit
%   Theorem, estimate the probability that the first strike achieves this
%   benchmark.\footnote{Responses deviating significantly from 100\% confidence
%     will be flagged for review by the Ministry of Truth.}

%   \textit{Hints:} The following may
%   be useful $\sqrt{48}= 4 \sqrt{3}$. Also, the graph of the standard
%   normal pdf is
%   \begin{center}
%     \includegraphics[scale=0.3]{images/standard-normal-small}
%   \end{center}
% \end{problem}

\end{document}