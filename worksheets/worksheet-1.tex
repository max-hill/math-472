\documentclass[10pt]{article}
\usepackage[margin=1in]{geometry}
\usepackage{graphicx}
\graphicspath{{./images/}}

\usepackage{amsmath, amssymb, amsthm, amsfonts, mathtools, bbm, breqn}
\usepackage[shortlabels]{enumitem}
\usepackage{multicol}       % multiple columns
\usepackage{abraces}        % asymmetric braces
\usepackage{skull}
\usepackage{tikz}
\usetikzlibrary{arrows.meta, arrows, calc, matrix, positioning}

\usepackage{hyperref}
\usepackage[capitalize]{cleveref}


% Citing theorems by name. (source: https://tex.stackexchange.com/questions/109843/cleveref-and-named-theorems)
\makeatletter
\newcommand{\ncref}[1]{\cref{#1}\mynameref{#1}{\csname r@#1\endcsname}}

\def\mynameref#1#2{%
  \begingroup
  \edef\@mytxt{#2}%
  \edef\@mytst{\expandafter\@thirdoffive\@mytxt}%
  \ifx\@mytst\empty\else
  \space(\nameref{#1})\fi
  \endgroup
}
\makeatother

% for the pipe symbol
\usepackage[T1]{fontenc}

\usepackage{xcolor} % Enables a broader range of colors

% Define custom colors
\definecolor{RoyalBlue}{cmyk}{1, 0.50, 0, 0}

% Note commands
\definecolor{Red}{rgb}{1,0,0}
\definecolor{Blue}{rgb}{0,0,1}
\definecolor{Purple}{rgb}{.75,0,.25}
\newcommand{\rnote}[1]{\textcolor{Red}{[#1]}}       % Red note
\newcommand{\pnote}[1]{\textcolor{Purple}{[#1]}}    % Purple note
\newcommand{\bnote}[1]{\textcolor{Blue}{#1}}        % Blue text

\usepackage{xparse} % allows paragraph-spaning color environment
\NewDocumentCommand{\Max}{+m}{%
  {\color{Red}#1}%
}

% Emphasized text
\newcommand{\demph}[1]{\textcolor{RoyalBlue}{\textbf{\slshape #1}}} % Slanted


% Claim numbering (the counter restarts after each proof environment)
\newcounter{claimcount}
\setcounter{claimcount}{0}
\newenvironment{claim}{\refstepcounter{claimcount}\par\addvspace{\medskipamount}\noindent\textbf{Claim \arabic{claimcount}:}}{}
\usepackage{etoolbox}
\AtBeginEnvironment{proof}{\setcounter{claimcount}{0}}
\newenvironment{claimproof}{\par\addvspace{\medskipamount}\noindent\textit{Proof of Claim  \arabic{claimcount}.}}{\hfill\ensuremath{\qedsymbol} \tiny{Claim}

  \medskip}
% Add claim support to cleverref
\crefname{claimcount}{Claim}{Claims}


% Math Environments
\newtheorem{theorem}{Theorem}
\newtheorem{assumption}[theorem]{Assumption}
\newtheorem{lemma}[theorem]{Lemma}
\newtheorem{proposition}[theorem]{Proposition}
\newtheorem{corollary}[theorem]{Corollary}
\newtheorem{question}[theorem]{Question}
\theoremstyle{definition}
\newtheorem{definition}[theorem]{Definition}
\newtheorem{remark}[theorem]{Remark}
\newtheorem{example}[theorem]{Example}
\newtheorem{notation}[theorem]{Notation}
\newtheorem{problem}[theorem]{Problem}

% Redefine the Example environment to include "End of example [number]"
\makeatletter
\let\oldexample\example
\renewenvironment{example}
{\begin{oldexample}}
  {\par\smallskip\hfill   End of Example~\theexample. $\square$    \par\end{oldexample}}
\makeatother

% Custom Math Commands
\newcommand{\vt}{\vspace{5mm}}                       % Vertical space
\newcommand{\fl}[1]{\noindent\textbf{#1}}            % Bold first line
\newcommand{\Fl}[1]{\vspace{5mm}\noindent\textbf{#1}}% Bold first line with space above
\newcommand{\norm}[1]{\left\lVert #1 \right\rVert}   % Norm

% Common math symbols
\newcommand{\R}{\mathbb{R}}           % Real numbers
\newcommand{\N}{\mathbb{N}}           % Natural numbers
\newcommand{\C}{\mathbb{C}}           % Complex numbers
\newcommand{\Z}{\mathbb{Z}}           % Integers
\newcommand{\Q}{\mathbb{Q}}           % Rational numbers
\newcommand{\E}{\mathbb{E}}           % Expectation
\renewcommand{\P}{\mathbb{P}}         % Probability (renamed to avoid \P clash)
\newcommand{\indep}{\perp\!\!\!\perp} % Independence symbol

% Operators
\newcommand{\Var}{\mathrm{Var}}   % Variance
\newcommand{\tr}{\mathrm{tr}}     % Trace
\newcommand{\dist}{\mathrm{dist}} % Distance


\usepackage{biblatex}
\addbibresource{refs.bib}



\begin{document}
\begin{center}
  \section*{Math 472: Worksheet 1}
  \subsection*{February 20, 2026}
\end{center}

\textit{\textbf{Instructions:} Please work on the problems in groups of 2 or
  3. You don't need to finish this today, but I will ask each group to submit
  solutions on February 27.}

\vspace{1em}

\begin{problem}
  \label{problem:chi-squared}
  This problem aims to help visualize an answer to the following question:
  \textit{``Suppose that our population of interest does not have a normal
    distribution. What does the sampling distribution of $\overline{Y}$ look
    like, and what is the effect of the sample size on the sampling
    distribution of $\overline{Y}$?''}

  For this problem we will assume the population is a chi-squared distribution
  with 3 degrees of freedom. You can visualize the shape of this distribution
  using the \texttt{R} code
\begin{verbatim}
# Draw a random sample of size n=1000000 from a chi-squared distribution with 3 degrees
# of freedom
y <- rchisq(1000000, df=3)
hist(y, probability = TRUE, breaks = 100)
\end{verbatim}
  Use \texttt{R} to complete the following.
  \begin{enumerate}[(a)]
    \item By drawing samples from the population and computing the sample mean
    and sample variance, estimate the mean $\mu$, variance $\sigma^{2}$, and
    standard deviation $\sigma$ of the population. (When $n$ is large, your
    answers will be close to the values from Proposition 60 in the lecture
    notes).
    \item We can visualize the sampling distribution of
    $\overline{Y}=(Y_{1}+\ldots+Y_{n})/n$ by generating a large number (say,
    $k=100,000$) of sample means for a specified value of $n$, and then
    plotting this histogram (use the setting \texttt{probability=TRUE} and
    specify a reasonable number of breaks).

    Do this for $n=4,16,32$ and $64$.
    
    To be extra spicy, you can overlay the normal curve approximation given by
    the central limit theorem by running the following code after you plot your
    histogram:

    \texttt{curve(dnorm(x, mean=3, sd=sqrt(6/n)), add=TRUE)}
    
    \item What do you notice about the adequacy of the normal approximation of
    the sampling distribution of $\overline{Y}$ as the sample size $n$
    increases?
    \item \label{item:1} What happens to the standard error of $\overline{Y}$ as $n$ goes
    from $4$ to $16$, from $16$ to $64$, and from $64$ to $256$? Formulate a general
    conjecture which relates the sample size to the standard error of
    $\overline{Y}$?  Can you prove your conjecture?
  \end{enumerate}
\end{problem}

% \begin{problem}
%   Do \cref{problem:chi-squared} again, but this time assume the population is
%   exponentially distributed with mean $2$. Does your conjecture from
%   \ref{item:1} hold up? Can you prove the conjecture?
% \end{problem}


\begin{problem}[1-dimensional Brownian motion]
  In this problem, we will model the price of a stock over the course of a
  day. Let $x(t)$ be the price of the stock at time $t$, where $t$ is
  measured in seconds. Assume that at the beginning of the day $(t=0)$, the
  stock trades at $\$5$, and that trading continues for 9 hours.

  We will model the volatility of the stock price in the following way. Assume
  that each second, the change in the stock prices is normally distributed
  with mean $\mu=0$ and standard deviation $\sigma = 1/180$. In other words,
  for every $n=0,1,2,\ldots$, we have
  \begin{equation}\label{eq:1}
    x(n+1)= x(n)+V_{n}
  \end{equation}
  where $V_{n}$ is an independent normal random variable with mean $\mu=0$ and
  standard deviation $\sigma=1/294$. We can think of $V_{1},V_{2},\ldots$ as
  representing the random \textit{volatility} of the stock price from moment
  to moment.

  
  \begin{enumerate}[(a)]
    \item Plot the stock price over the course of the day. (You can use the
    command \texttt{plot()} to do this, and I recommend using the optional
    argument \texttt{type="l"} to tell \texttt{R} to draw connected lines.)
    \item \label{item:2} Let $Y$ be the change in the price of the stock
    (i.e., the final price minus the initial price). Generate a sample of
    $n=5000$ values of $Y$, which you should save as the vector \texttt{y_values}.

    (Hint: one way to do this is to first write a function which returns $Y$,
    and then use the \texttt{replicate()} command.)
    \item Compute the sample mean and sample standard deviation of \texttt{y_values}.
    \item Plot a histogram of your sample from \ref{item:2} using
    \begin{center}
      \texttt{hist(y_values, probability=TRUE, breaks=100)}
    \end{center}
    After doing this, overlay a normal curve on top of your histogram with the command
    \begin{center}
      \texttt{curve(dnorm(x,mean=??,sd=??), add=true)}
    \end{center}
    (where you have to substitute sensible values for the mean and standard deviation).
    
    \item Prove that $Y$ has a standard normal distribution.

    (Hint: use \cref{eq:1}).
    \item Suppose you purchase the stock at the beginning of the day and sell
    at the end. What is the probability that you make at least $\$1$?
  \end{enumerate}
\end{problem}

\begin{problem}[2-dimensional Brownian motion]
  Suppose we track the motion of a pollen grain suspended in a petri dish
  under microscope. The pollen grain is constantly being bombarded by water
  molecules from every direction, with each collision transferring a small
  amount of momentum to the pollen grain. This causes the pollen grain to
  slowly move around in an random, irregular way.

  Let $(x(t),y(t))$ be the position of the pollen grain after
  $t\in \mathbb{N}$ units of time. Assume that at time $t=0$, the pollen grain
  starts at position $(x(0),y(0))=(0,0)$, and that at each subsequent time
  step, its position is given by
  \begin{equation*}
    (x(t+1),y(t+1)) = (x(t)+X_{t}, y(t)+Y_{t}),
  \end{equation*}
  where $X_{1},X_{2},\ldots,$ and $Y_{1},Y_{2}\ldots$ are independent normal
  random variables with mean 0 and variance $\sigma^{2}>0$.

  \begin{enumerate}[(a)]
    \item Assume that $\sigma=1/10$. Plot the motion of the pollen grain from
    time $t=0$ to $t=100$.

    Hint: One way to do this is generate a vector of 100 $x$-coordinates and a
    vector of 100 $y$-coordinates independently, and then plot them using the
    following command:
    
    \begin{center}
      \texttt{plot(x,y, pch=16, type="l")}
    \end{center}
    
    \item \label{item:3} Let $D$ be the distance of the pollen grain from the
    origin after $100$ times steps (we are still assuming $\sigma=1/10$).
    Generate a large number of independent samples of $D$. Use them to
    estimate the mean and variance of $D$, and plot a histogram of your $D$
    values.

    \item Now, assume that $\sigma=1/100$. Plot the motion of the pollen grain from
    time $t=0$ to $t=10000$.

    \item \label{item:4} With $\sigma=1/100$, let $D$ be the distance of the pollen grain
    from the origin after $1000$. Generate a large number of independent
    samples of $D$. Use them to estimate the mean and variance of $D$, and
    plot a histogram of your $D$ values.
    
    \item Compare the distributions of \ref{item:3} and \ref{item:4}. Does the
    distribution obtained in change substantially?
    
    \item What is the distribution of $D^{2}$ in both cases?
  \end{enumerate}
\end{problem}
\printbibliography
\end{document}